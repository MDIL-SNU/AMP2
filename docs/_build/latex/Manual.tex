%% Generated by Sphinx.
\def\sphinxdocclass{report}
\documentclass[letterpaper,10pt,english]{sphinxmanual}
\ifdefined\pdfpxdimen
   \let\sphinxpxdimen\pdfpxdimen\else\newdimen\sphinxpxdimen
\fi \sphinxpxdimen=.75bp\relax

\PassOptionsToPackage{warn}{textcomp}
\usepackage[utf8]{inputenc}
\ifdefined\DeclareUnicodeCharacter
% support both utf8 and utf8x syntaxes
  \ifdefined\DeclareUnicodeCharacterAsOptional
    \def\sphinxDUC#1{\DeclareUnicodeCharacter{"#1}}
  \else
    \let\sphinxDUC\DeclareUnicodeCharacter
  \fi
  \sphinxDUC{00A0}{\nobreakspace}
  \sphinxDUC{2500}{\sphinxunichar{2500}}
  \sphinxDUC{2502}{\sphinxunichar{2502}}
  \sphinxDUC{2514}{\sphinxunichar{2514}}
  \sphinxDUC{251C}{\sphinxunichar{251C}}
  \sphinxDUC{2572}{\textbackslash}
\fi
\usepackage{cmap}
\usepackage[T1]{fontenc}
\usepackage{amsmath,amssymb,amstext}
\usepackage{babel}



\usepackage{times}
\expandafter\ifx\csname T@LGR\endcsname\relax
\else
% LGR was declared as font encoding
  \substitutefont{LGR}{\rmdefault}{cmr}
  \substitutefont{LGR}{\sfdefault}{cmss}
  \substitutefont{LGR}{\ttdefault}{cmtt}
\fi
\expandafter\ifx\csname T@X2\endcsname\relax
  \expandafter\ifx\csname T@T2A\endcsname\relax
  \else
  % T2A was declared as font encoding
    \substitutefont{T2A}{\rmdefault}{cmr}
    \substitutefont{T2A}{\sfdefault}{cmss}
    \substitutefont{T2A}{\ttdefault}{cmtt}
  \fi
\else
% X2 was declared as font encoding
  \substitutefont{X2}{\rmdefault}{cmr}
  \substitutefont{X2}{\sfdefault}{cmss}
  \substitutefont{X2}{\ttdefault}{cmtt}
\fi


\usepackage[Bjarne]{fncychap}
\usepackage{sphinx}

\fvset{fontsize=\small}
\usepackage{geometry}

% Include hyperref last.
\usepackage{hyperref}
% Fix anchor placement for figures with captions.
\usepackage{hypcap}% it must be loaded after hyperref.
% Set up styles of URL: it should be placed after hyperref.
\urlstyle{same}
\addto\captionsenglish{\renewcommand{\contentsname}{Contents:}}

\usepackage{sphinxmessages}
\setcounter{tocdepth}{3}
\setcounter{secnumdepth}{3}


\title{AMP$^\text{2}$ Manual \\ \textit{0.9.5}}
\date{Feb 28, 2020}
\release{}
\author{Yong Youn \\ Miso Lee \\ Changho Hong \\ Doyeon Kim \\ Sangtae Kim \\ Jisu Jung \\ Kanghoon Yim \\ Seungwu Han}
\newcommand{\sphinxlogo}{\vbox{}}
\renewcommand{\releasename}{}
\makeindex
\begin{document}

\ifdefined\shorthandoff
  \ifnum\catcode`\=\string=\active\shorthandoff{=}\fi
  \ifnum\catcode`\"=\active\shorthandoff{"}\fi
\fi

\pagestyle{empty}
\sphinxmaketitle
\pagestyle{plain}
\sphinxtableofcontents
\pagestyle{normal}
\phantomsection\label{\detokenize{index::doc}}



\chapter{Installation and execution}
\label{\detokenize{Installation/Installation:installation-and-execution}}\label{\detokenize{Installation/Installation::doc}}

\section{Installation AMP$^{\text{2}}$}
\label{\detokenize{Installation/Installation:installation-amp2}}

\subsection{System requirements}
\label{\detokenize{Installation/Installation:system-requirements}}\begin{quote}

AMP$^{\text{2}}$ supports Python 2.7 and 3. Currently, the package is not compatible
with lower version than 2.7. AMP$^{\text{2}}$ is utilizes Python modules
in the following with link to each site.
\begin{itemize}
\item {} 
numpy {[}\sphinxurl{https://www.numpy.org}{]}

\item {} 
scipy {[}\sphinxurl{https://www.scipy.org}{]}

\item {} 
spglib {[}\sphinxurl{https://atztogo.github.io/spglib}{]}

\item {} 
PyYAML {[}\sphinxurl{https://pypi.org/project/PyYAML}{]}

\end{itemize}

These modules should be pre-installed. In addition, AMP$^{\text{2}}$ needs gnuplot to draw
various figures.
\end{quote}


\subsection{Installation}
\label{\detokenize{Installation/Installation:installation}}\begin{quote}

To use AMP$^{\text{2}}$, please download the file from \sphinxurl{https://github.com/MDIL-SNU/AMP2} under the
working directory.
\end{quote}


\section{Essential setting}
\label{\detokenize{Installation/Installation:essential-setting}}

\subsection{Setting of configuration file}
\label{\detokenize{Installation/Installation:setting-of-configuration-file}}\begin{quote}

AMP$^{\text{2}}$ uses YAML style configuration file. All setting parameters used in AMP$^{\text{2}}$ can
be controlled in “config.yaml”. Before using AMP$^{\text{2}}$, proper pathes and mpi program command
should be set to be suitable for your system. Following commands are the essential directories
and programs to be set.

\begin{sphinxVerbatim}[commandchars=\\\{\}]
\PYG{n}{Directory}\PYG{p}{:}
  \PYG{n}{submit}\PYG{p}{:}
  \PYG{n}{src}\PYG{p}{:}
  \PYG{n}{pot\PYGZus{}path\PYGZus{}gga}\PYG{p}{:}
  \PYG{n}{pot\PYGZus{}path\PYGZus{}lda}\PYG{p}{:}
\PYG{n}{Program}\PYG{p}{:}
  \PYG{n}{vasp\PYGZus{}std}\PYG{p}{:}
  \PYG{n}{vasp\PYGZus{}gam}\PYG{p}{:}
  \PYG{n}{vasp\PYGZus{}ncl}\PYG{p}{:}
  \PYG{n}{gnuplot}\PYG{p}{:}
  \PYG{n}{mpi\PYGZus{}command}\PYG{p}{:}
\end{sphinxVerbatim}

Details for the commands are in {\hyperref[\detokenize{Input/Configuration::doc}]{\sphinxcrossref{\DUrole{doc}{Configuration}}}}.
\end{quote}


\section{Execution AMP$^{\text{2}}$}
\label{\detokenize{Installation/Installation:execution-amp2}}
You can execute AMP$^{\text{2}}$ using Python command as following.

\begin{sphinxVerbatim}[commandchars=\\\{\}]
\PYG{n}{python} \PYG{p}{[}\PYG{n}{src\PYGZus{}path}\PYG{p}{]}\PYG{o}{/}\PYG{n}{main}\PYG{o}{.}\PYG{n}{py} \PYG{p}{[}\PYG{n}{path} \PYG{k}{for} \PYG{n}{configuration} \PYG{n}{file}\PYG{p}{]} \PYG{p}{[}\PYG{n}{path} \PYG{k}{for} \PYG{n}{nodefile}\PYG{p}{]} \PYG{p}{[}\PYG{n}{the} \PYG{n}{number} \PYG{n}{of} \PYG{n}{cores}\PYG{p}{]}
\end{sphinxVerbatim}
\begin{itemize}
\item {} 
{[}src\_path{]} is the path for directory of source codes for AMP$^{\text{2}}$.

\item {} 
{[}path for configuration file{]} is the path for configuration file (config.yaml).

\item {} 
{[}path for nodefile{]} is used to record the information of computing nodes such as PBS\_nodefile in
Portable Batch System (PBS) and HOSTNAME in Sun Grid Engine (SGE). In the PBS system, we recommand to use
the command, “echo \$PBS\_nodefile \textgreater{} nodefile”. Also, users can save an arbitrary text
by writing in the nodefile.

\item {} 
{[}the number of cores{]} is the number of cores to be used in parallel computing.

\end{itemize}

For the convenience, we provide the shell script file (run.sh) as following.

\begin{sphinxVerbatim}[commandchars=\\\{\}]
echo \PYGZsq{}node information\PYGZsq{} \PYGZgt{} nodefile
NPROC=16       \PYGZsh{} The number of cores for parallel computing

\PYGZsh{}\PYGZsh{}\PYGZsh{} set path of config.yaml \PYGZsh{}\PYGZsh{}\PYGZsh{}
conf=./config.yaml
\PYGZsh{}\PYGZsh{}\PYGZsh{}\PYGZsh{}\PYGZsh{}\PYGZsh{}\PYGZsh{}\PYGZsh{}\PYGZsh{}\PYGZsh{}\PYGZsh{}\PYGZsh{}\PYGZsh{}\PYGZsh{}\PYGZsh{}\PYGZsh{}\PYGZsh{}\PYGZsh{}\PYGZsh{}\PYGZsh{}\PYGZsh{}\PYGZsh{}\PYGZsh{}\PYGZsh{}\PYGZsh{}\PYGZsh{}\PYGZsh{}\PYGZsh{}\PYGZsh{}\PYGZsh{}\PYGZsh{}

\PYGZsh{}\PYGZsh{}\PYGZsh{} Do not change \PYGZsh{}\PYGZsh{}\PYGZsh{}\PYGZsh{}\PYGZsh{}\PYGZsh{}\PYGZsh{}\PYGZsh{}\PYGZsh{}\PYGZsh{}\PYGZsh{}\PYGZsh{}\PYGZsh{}
src\PYGZus{}path={}`grep \PYGZsq{}src\PYGZus{}path\PYGZsq{} \PYGZdl{}conf \textbar{} tr \PYGZhy{}s \PYGZsq{} \PYGZsq{} \textbar{} cut \PYGZhy{}d \PYGZdq{} \PYGZdq{} \PYGZhy{}f 3{}`
\PYGZsh{}\PYGZsh{}\PYGZsh{}\PYGZsh{}\PYGZsh{}\PYGZsh{}\PYGZsh{}\PYGZsh{}\PYGZsh{}\PYGZsh{}\PYGZsh{}\PYGZsh{}\PYGZsh{}\PYGZsh{}\PYGZsh{}\PYGZsh{}\PYGZsh{}\PYGZsh{}\PYGZsh{}\PYGZsh{}\PYGZsh{}\PYGZsh{}\PYGZsh{}\PYGZsh{}\PYGZsh{}\PYGZsh{}\PYGZsh{}\PYGZsh{}\PYGZsh{}\PYGZsh{}\PYGZsh{}

python \PYGZdl{}src\PYGZus{}path/main.py \PYGZdl{}conf nodefile \PYGZdl{}NPROC \PYGZgt{}\PYGZam{} stdout.x
\end{sphinxVerbatim}

Before execution, you need to modify \sphinxstyleemphasis{‘node information’}, \sphinxstyleemphasis{NPROC} and \sphinxstyleemphasis{conf}.
Then, you can execute AMP$^{\text{2}}$ using shell script as following.

\begin{sphinxVerbatim}[commandchars=\\\{\}]
\PYG{n}{sh} \PYG{n}{run}\PYG{o}{.}\PYG{n}{sh}
\end{sphinxVerbatim}

The shell script file can be easily integrated with job scheduler program such
as PBS.


\chapter{Overview}
\label{\detokenize{Overview/Overview:overview}}\label{\detokenize{Overview/Overview::doc}}

\section{Preparing input files}
\label{\detokenize{Overview/Overview:preparing-input-files}}
Before running AMP$^{\text{2}}$, two input files should be prepared such as YAML style configuration
file (config.yaml) and structure file. The details for input files are explained in {\hyperref[\detokenize{Input/Input_files::doc}]{\sphinxcrossref{\DUrole{doc}{Input files}}}}.
The basic format of config.yaml and structure files are like below:
\begin{quote}

config.yaml:

\begin{sphinxVerbatim}[commandchars=\\\{\}]
\PYG{n}{directory}\PYG{p}{:}
  \PYG{n}{submit}\PYG{p}{:} \PYG{o}{.}\PYG{o}{/}\PYG{n}{Submit}                      \PYG{c+c1}{\PYGZsh{} the path of structure file or the directory containg structure files}
  \PYG{n}{output}\PYG{p}{:} \PYG{o}{.}\PYG{o}{/}\PYG{n}{Output}                      \PYG{c+c1}{\PYGZsh{} the path of the directory where calculation is conducted}
  \PYG{n}{done}\PYG{p}{:} \PYG{o}{.}\PYG{o}{/}\PYG{n}{Done}                          \PYG{c+c1}{\PYGZsh{} the path of the directory where results are saved}
  \PYG{n}{error}\PYG{p}{:} \PYG{o}{.}\PYG{o}{/}\PYG{n}{ERROR}                        \PYG{c+c1}{\PYGZsh{} the path of the directory where the materials with error are saved}
  \PYG{n}{src\PYGZus{}path}\PYG{p}{:} \PYG{o}{.}\PYG{o}{/}\PYG{n}{src}                       \PYG{c+c1}{\PYGZsh{} the path of the directory of AMP2 source codes}
  \PYG{n}{pot\PYGZus{}path\PYGZus{}gga}\PYG{p}{:} \PYG{o}{.}\PYG{o}{/}\PYG{n}{pot}\PYG{o}{/}\PYG{n}{PBE}               \PYG{c+c1}{\PYGZsh{} the path of directory for GGA pseudopotential}
  \PYG{n}{pot\PYGZus{}path\PYGZus{}lda}\PYG{p}{:} \PYG{o}{.}\PYG{o}{/}\PYG{n}{pot}\PYG{o}{/}\PYG{n}{LDA}               \PYG{c+c1}{\PYGZsh{} the path of directory for LDA pseudopotential}

\PYG{n}{program}\PYG{p}{:}
  \PYG{n}{vasp\PYGZus{}std}\PYG{p}{:} \PYG{o}{.}\PYG{o}{/}\PYG{n}{vasp\PYGZus{}std}                  \PYG{c+c1}{\PYGZsh{} the path of standard version of VASP}
  \PYG{n}{vasp\PYGZus{}gam}\PYG{p}{:} \PYG{o}{.}\PYG{o}{/}\PYG{n}{vasp\PYGZus{}gam}                  \PYG{c+c1}{\PYGZsh{} the path of gamma\PYGZhy{}only version of VASP}
  \PYG{n}{vasp\PYGZus{}ncl}\PYG{p}{:} \PYG{o}{.}\PYG{o}{/}\PYG{n}{vasp\PYGZus{}ncl}                  \PYG{c+c1}{\PYGZsh{} the path of noncollinear version of VASP}
  \PYG{n}{gnuplot}\PYG{p}{:} \PYG{o}{/}\PYG{n}{usr}\PYG{o}{/}\PYG{n}{local}\PYG{o}{/}\PYG{n+nb}{bin}\PYG{o}{/}\PYG{n}{gnuplot}       \PYG{c+c1}{\PYGZsh{} the path of executable file for gnuplot}
  \PYG{n}{mpi\PYGZus{}command}\PYG{p}{:} \PYG{n}{mpirun}                   \PYG{c+c1}{\PYGZsh{} mpi command (ex. mpirun, mpiexec, ...)}

\PYG{n}{vasp\PYGZus{}parallel}\PYG{p}{:}
  \PYG{n}{npar}\PYG{p}{:} \PYG{l+m+mi}{2}                               \PYG{c+c1}{\PYGZsh{} the number of bands that are treated in parallel. It is same to NPAR tag in VASP.}
  \PYG{n}{kpar}\PYG{p}{:} \PYG{l+m+mi}{2}                               \PYG{c+c1}{\PYGZsh{} the number of kpoints that are treated in parallel. It is same to NPAR tag in VASP.}
\end{sphinxVerbatim}

Structure file (VASP structure file format):

\begin{sphinxVerbatim}[commandchars=\\\{\}]
Primitive Cell
   1.000000000
      0.0    2.714895    2.714895
      2.714895    0.0    2.714895
      2.714895    2.714895    0.0
    Si
    2
Selective dynamics
Direct
    0.5    0.5    0.5  T  T  T ! Si1
    0.75    0.75    0.75  T  T  T ! Si1
\end{sphinxVerbatim}
\end{quote}


\section{Running AMP$^{\text{2}}$}
\label{\detokenize{Overview/Overview:running-amp2}}
You can execute AMP$^{\text{2}}$ using shell script as following.

\begin{sphinxVerbatim}[commandchars=\\\{\}]
\PYG{n}{sh} \PYG{n}{run}\PYG{o}{.}\PYG{n}{sh}
\end{sphinxVerbatim}

The details for shell script are mentioned in the section, “Execution AMP$^{\text{2}}$” in {\hyperref[\detokenize{Installation/Installation::doc}]{\sphinxcrossref{\DUrole{doc}{Installation and execution}}}}.


\section{Outputs}
\label{\detokenize{Overview/Overview:outputs}}
After starting the calculation, new directory is formed in \sphinxstyleemphasis{output\_path} as the name of the structure
file. (\sphinxstyleemphasis{name} directory is formed from \sphinxstyleemphasis{name.cif} or \sphinxstyleemphasis{POSCAR\_name}.)
Then, if calculation is well finished, the directory moves to \sphinxstyleemphasis{done\_path}. If not, it moves to \sphinxstyleemphasis{error\_path}.
The following data are the examples of calculation results for Cr$_{\text{2}}$O$_{\text{3}}$.
More details for output files are written in {\hyperref[\detokenize{Output/Output::doc}]{\sphinxcrossref{\DUrole{doc}{Output}}}}.
\begin{quote}

POSCAR\_GGA:

\begin{sphinxVerbatim}[commandchars=\\\{\}]
relaxed poscar
1.000000000
    2.53085784423    1.46119145764    4.60391533726
    \PYGZhy{}2.53085784423    1.46119145764    4.60391533726
    0.0    \PYGZhy{}2.9223829153    4.60391533726
    Cr    O
    4    6
Selective dynamics
Direct
    0.348055231569    0.348055231569    0.348055231569  T  T  T ! Cr1\PYGZus{}up
    0.848055231569    0.848055231569    0.848055231569  T  T  T ! Cr1\PYGZus{}up
    0.151944768431    0.151944768431    0.151944768431  T  T  T ! Cr1\PYGZus{}down
    0.651944768431    0.651944768431    0.651944768431  T  T  T ! Cr1\PYGZus{}down
    0.553903778143    0.946096221857    0.25  T  T  T ! O1
    0.946096221857    0.25    0.553903778143  T  T  T ! O1
    0.25    0.553903778143    0.946096221857  T  T  T ! O1
    0.0539037781426    0.75    0.446096221857  T  T  T ! O1
    0.75    0.446096221857    0.0539037781426  T  T  T ! O1
    0.446096221857    0.0539037781426    0.75  T  T  T ! O1
\end{sphinxVerbatim}

Band\_gap\_GGA.log:

\begin{sphinxVerbatim}[commandchars=\\\{\}]
\PYG{n}{Band} \PYG{n}{gap}\PYG{p}{:}      \PYG{l+m+mf}{2.734} \PYG{n}{eV} \PYG{p}{(}\PYG{n}{Indirect}\PYG{p}{)}

\PYG{n}{VBM}\PYG{p}{:} \PYG{l+m+mf}{0.2916667}  \PYG{l+m+mf}{0.0}  \PYG{l+m+mf}{0.0}   \PYG{p}{:}      \PYG{l+m+mf}{3.366} \PYG{n}{eV}
\PYG{n}{CBM}\PYG{p}{:} \PYG{l+m+mf}{0.42206}  \PYG{l+m+mf}{0.42206}  \PYG{o}{\PYGZhy{}}\PYG{l+m+mf}{0.01078659}   \PYG{p}{:}      \PYG{l+m+mf}{6.100} \PYG{n}{eV}

\PYG{n}{nVBM}\PYG{p}{:} \PYG{l+m+mi}{30}  \PYG{n}{spin}\PYG{p}{:} \PYG{l+m+mi}{1}
\PYG{n}{nCBM}\PYG{p}{:} \PYG{l+m+mi}{31}  \PYG{n}{spin}\PYG{p}{:} \PYG{l+m+mi}{1}
\end{sphinxVerbatim}

band\_GGA.png:
\begin{quote}

\noindent\sphinxincludegraphics[width=300\sphinxpxdimen]{{band_GGA3}.png}
\end{quote}

dos\_GGA.png:
\begin{quote}

\noindent\sphinxincludegraphics[width=150\sphinxpxdimen]{{dos_GGA2}.png}
\end{quote}
\end{quote}


\section{List of source codes}
\label{\detokenize{Overview/Overview:list-of-source-codes}}
AMP$^{\text{2}}$ consists of several python codes as follows:
\begin{itemize}
\item {} \begin{description}
\item[{main.py:}] \leavevmode
This is main code to run AMP$^{\text{2}}$.

\end{description}

\item {} \begin{description}
\item[{amp2\_input.py:}] \leavevmode
This is for generating input files for VASP from structure file.

\end{description}

\item {} \begin{description}
\item[{kpoint.py:}] \leavevmode
This is for conducting a convergence test of k-points.

\end{description}

\item {} \begin{description}
\item[{cutoff.py:}] \leavevmode
This is for conducting a convergence test of cutoff energy.

\end{description}

\item {} \begin{description}
\item[{relax.py:}] \leavevmode
This is for conducting structure optimization.

\end{description}

\item {} \begin{description}
\item[{magnetic\_ordering.py:}] \leavevmode
This is for identifying the most stable magnetic spin ordering.

\end{description}

\item {} \begin{description}
\item[{band.py:}] \leavevmode
This is for drawing band structure and estimating band gap.

\end{description}

\item {} \begin{description}
\item[{dos.py:}] \leavevmode
This is for drawing density of states.

\end{description}

\item {} \begin{description}
\item[{hse\_gap.py:}] \leavevmode
This is for estimating band gap with \sphinxhref{mailto:PBE@HSE}{PBE@HSE} scheme.

\end{description}

\item {} \begin{description}
\item[{effm.py:}] \leavevmode
This is for estimating effective masses of hole and electron.

\end{description}

\item {} \begin{description}
\item[{dielectric.py:}] \leavevmode
This is for estimating dielectric tensor.

\end{description}

\item {} \begin{description}
\item[{get\_result.py:}] \leavevmode
This is for summarizing the calculation results.

\end{description}

\item {} \begin{description}
\item[{input\_conf.py:}] \leavevmode
This is for handling YAML type configuration.

\end{description}

\item {} \begin{description}
\item[{rerun\_for\_metal.py:}] \leavevmode
This is a code to restart the all calculations without the on-site Uterm if the material was found to be metallic and Uwas applied.

\end{description}

\item {} \begin{description}
\item[{genetic\_algorithm.py:}] \leavevmode
This is for performing genetic algorithm to find the most stable magnetic
spin ordering.

\end{description}

\item {} \begin{description}
\item[{genetic\_operator.py:}] \leavevmode
This is a package of modules for performing genetic algorithm.

\end{description}

\item {} \begin{description}
\item[{make\_supercell.py:}] \leavevmode
This is a code to build supercell to find magnetic primitive cell.

\end{description}

\item {} \begin{description}
\item[{mk\_suprecell.py:}] \leavevmode
This is a code to build supercell for the Ising coefficient.

\end{description}

\item {} \begin{description}
\item[{z\_subr.py:}] \leavevmode
This is a package of modules for ’mk\_supercell.py’.

\end{description}

\item {} \begin{description}
\item[{module\_amp2\_input.py:}] \leavevmode
This is a package of modules for generating input files for VASP from structure file.

\end{description}

\item {} 
module\_converge.py: This is a package of modules for convergence test.

\item {} \begin{description}
\item[{module\_relax.py:}] \leavevmode
This is a package of modules for structure optimization.

\end{description}

\item {} \begin{description}
\item[{module\_AF.py:}] \leavevmode
This is a package of modules for identifying the most stable magnetic spin ordering.

\end{description}

\item {} \begin{description}
\item[{module\_GA.py:}] \leavevmode
This is a package of modules for genetic algorithm.

\end{description}

\item {} \begin{description}
\item[{module\_band.py:}] \leavevmode
This is a package of modules for drawing band structure and calculating band gap.

\end{description}

\item {} \begin{description}
\item[{module\_dos.py:}] \leavevmode
This is a package of modules for drawing density of states.

\end{description}

\item {} \begin{description}
\item[{module\_hse.py:}] \leavevmode
This is a package of modules for calculating band gap with \sphinxhref{mailto:HSE@PBE}{HSE@PBE} scheme.

\end{description}

\item {} \begin{description}
\item[{module\_effm.py:}] \leavevmode
This is a package of modules for calculating effective mass.

\end{description}

\item {} \begin{description}
\item[{module\_dielectric.py:}] \leavevmode
This is a package of modules for calculating dielectric tensor.

\end{description}

\item {} \begin{description}
\item[{module\_vasprun.py:}] \leavevmode
This is a package of modules to run VASP.

\end{description}

\item {} \begin{description}
\item[{module\_log.py:}] \leavevmode
This is a package of modules to record log.

\end{description}

\item {} \begin{description}
\item[{module\_vector.py:}] \leavevmode
This is a package of modules to calculate several properties such as distance between two points and angle.

\end{description}

\end{itemize}

Additionally, there are files for predefined variables.
\begin{itemize}
\item {} \begin{description}
\item[{INCAR0:}] \leavevmode
This is for default configuration for ’INCAR’.

\end{description}

\item {} \begin{description}
\item[{U\_table.yaml:}] \leavevmode
This is for default Uparameters.

\end{description}

\item {} \begin{description}
\item[{pot\_table.yaml:}] \leavevmode
This is for default potential files.

\end{description}

\item {} \begin{description}
\item[{config\_def.yaml:}] \leavevmode
This is default configuration for ’config.yaml’.

\end{description}

\end{itemize}


\chapter{Input}
\label{\detokenize{Input/Input:input}}\label{\detokenize{Input/Input::doc}}

\section{Input files}
\label{\detokenize{Input/Input_files:input-files}}\label{\detokenize{Input/Input_files::doc}}

\subsection{Structure file}
\label{\detokenize{Input/Input_files:structure-file}}\begin{quote}

The valid formats for structure file are that for VASP and cif format. In the cif files,
symmetry operator (\_space\_group\_symop\_{[}{]} or \_symmetry\_equiv\_{[}{]}), atomic label (\_atom\_site\_label),
occupancy (\_atom\_site\_occupancy) and fractional positions (\_atom\_site\_fract\_{[}{]}) must be included.
The name of structure files must be formatted as name.cif or POSCAR\_name where tag is used for identification.

VASP structure file format:

\begin{sphinxVerbatim}[commandchars=\\\{\}]
Primitive Cell
   1.000000000
      0.0    2.714895    2.714895
      2.714895    0.0    2.714895
      2.714895    2.714895    0.0
    Si
    2
Selective dynamics
Direct
    0.5    0.5    0.5  T  T  T ! Si1
    0.75    0.75    0.75  T  T  T ! Si1
\end{sphinxVerbatim}
\end{quote}


\subsection{Configuration}
\label{\detokenize{Input/Input_files:configuration}}\begin{quote}

All of parameters can be tuned in the configuration file as following.
The detail for each parameter is explained in {\hyperref[\detokenize{Input/Configuration::doc}]{\sphinxcrossref{\DUrole{doc}{Configuration}}}}.

config.yaml:

\begin{sphinxVerbatim}[commandchars=\\\{\}]
\PYG{n}{directory}\PYG{p}{:}
  \PYG{n}{submit}\PYG{p}{:} \PYG{o}{.}\PYG{o}{/}\PYG{n}{Submit}                      \PYG{c+c1}{\PYGZsh{} the path of structure file or the directory containg structure files}
  \PYG{n}{output}\PYG{p}{:} \PYG{o}{.}\PYG{o}{/}\PYG{n}{Output}                      \PYG{c+c1}{\PYGZsh{} the path of the directory where calculation is conducted}
  \PYG{n}{done}\PYG{p}{:} \PYG{o}{.}\PYG{o}{/}\PYG{n}{Done}                          \PYG{c+c1}{\PYGZsh{} the path of the directory where results are saved}
  \PYG{n}{error}\PYG{p}{:} \PYG{o}{.}\PYG{o}{/}\PYG{n}{ERROR}                        \PYG{c+c1}{\PYGZsh{} the path of the directory where the materials with error are saved}
  \PYG{n}{src\PYGZus{}path}\PYG{p}{:} \PYG{o}{.}\PYG{o}{/}\PYG{n}{src}                       \PYG{c+c1}{\PYGZsh{} the path of the directory of AMP2 source codes}
  \PYG{n}{pot\PYGZus{}path\PYGZus{}gga}\PYG{p}{:} \PYG{o}{.}\PYG{o}{/}\PYG{n}{pot}\PYG{o}{/}\PYG{n}{PBE}               \PYG{c+c1}{\PYGZsh{} the path of directory for GGA pseudopotential}
  \PYG{n}{pot\PYGZus{}path\PYGZus{}lda}\PYG{p}{:} \PYG{o}{.}\PYG{o}{/}\PYG{n}{pot}\PYG{o}{/}\PYG{n}{LDA}               \PYG{c+c1}{\PYGZsh{} the path of directory for LDA pseudopotential}

\PYG{n}{program}\PYG{p}{:}
  \PYG{n}{vasp\PYGZus{}std}\PYG{p}{:} \PYG{o}{.}\PYG{o}{/}\PYG{n}{vasp\PYGZus{}std}                  \PYG{c+c1}{\PYGZsh{} the path of standard version of VASP}
  \PYG{n}{vasp\PYGZus{}gam}\PYG{p}{:} \PYG{o}{.}\PYG{o}{/}\PYG{n}{vasp\PYGZus{}gam}                  \PYG{c+c1}{\PYGZsh{} the path of gamma\PYGZhy{}only version of VASP}
  \PYG{n}{vasp\PYGZus{}ncl}\PYG{p}{:} \PYG{o}{.}\PYG{o}{/}\PYG{n}{vasp\PYGZus{}ncl}                  \PYG{c+c1}{\PYGZsh{} the path of noncollinear version of VASP}
  \PYG{n}{gnuplot}\PYG{p}{:} \PYG{o}{/}\PYG{n}{gnuplot}                     \PYG{c+c1}{\PYGZsh{} the path of executable file for gnuplot}
  \PYG{n}{mpi\PYGZus{}command}\PYG{p}{:} \PYG{n}{mpirun}                   \PYG{c+c1}{\PYGZsh{} mpi command (ex. mpirun, mpiexec, ...)}

\PYG{n}{calculation}\PYG{p}{:}
  \PYG{n}{magnetic\PYGZus{}ordering}\PYG{p}{:} \PYG{n}{T}                  \PYG{c+c1}{\PYGZsh{} On/Off for the calculation to idetify most stable magnetic spin ordering}
  \PYG{n}{band}\PYG{p}{:} \PYG{n}{T}                               \PYG{c+c1}{\PYGZsh{} On/Off for the calculation for band structure and band gap}
  \PYG{n}{density\PYGZus{}of\PYGZus{}states}\PYG{p}{:} \PYG{n}{T}                  \PYG{c+c1}{\PYGZsh{} On/Off for the calculation for density of states}
  \PYG{n}{hse\PYGZus{}oneshot}\PYG{p}{:} \PYG{n}{T}                        \PYG{c+c1}{\PYGZsh{} On/Off for the calculation for HSE@PBE}
  \PYG{n}{dielectric}\PYG{p}{:} \PYG{n}{T}                         \PYG{c+c1}{\PYGZsh{} On/Off for the calculation for dielectric constant}
  \PYG{n}{effective\PYGZus{}mass}\PYG{p}{:} \PYG{n}{T}                     \PYG{c+c1}{\PYGZsh{} On/Off for the calculation for effective mass}
  \PYG{n}{potential\PYGZus{}type}\PYG{p}{:} \PYG{n}{GGA}                   \PYG{c+c1}{\PYGZsh{} calculation scheme (LDA or GGA)}

\PYG{n}{vasp\PYGZus{}parallel}\PYG{p}{:}
  \PYG{n}{npar}\PYG{p}{:} \PYG{l+m+mi}{2}                               \PYG{c+c1}{\PYGZsh{} the number of bands that are treated in parallel. It is same to NPAR tag in VASP.}
  \PYG{n}{kpar}\PYG{p}{:} \PYG{l+m+mi}{2}                               \PYG{c+c1}{\PYGZsh{} the number of kpoints that are treated in parallel. It is same to NPAR tag in VASP.}

\PYG{n}{cif2vasp}\PYG{p}{:}
  \PYG{n}{pot\PYGZus{}name}\PYG{p}{:}                             \PYG{c+c1}{\PYGZsh{} the pseudopotential potential for element.}
    \PYG{n}{GGA}\PYG{p}{:}                                \PYG{c+c1}{\PYGZsh{} (Ex. GGA:\PYGZbs{}n    Ge:Ge\PYGZus{}d\PYGZbs{}n    Cu:Cu\PYGZus{}pv)}
    \PYG{n}{LDA}\PYG{p}{:}
  \PYG{n}{soc\PYGZus{}target}\PYG{p}{:}                           \PYG{c+c1}{\PYGZsh{} the elements to carry out spin\PYGZhy{}orbit coupling calculation (Ex. soc\PYGZus{}target:\PYGZbs{}n    \PYGZhy{} Bi\PYGZbs{}n    \PYGZhy{}Pb)}
  \PYG{n}{u\PYGZus{}value}\PYG{p}{:}                              \PYG{c+c1}{\PYGZsh{} U values for PBE+U calculation (Ex. u\PYGZus{}value:\PYGZbs{}n    La: 7.5\PYGZbs{}n    Ce: 8.5)}

\PYG{n}{hybrid\PYGZus{}oneshot}\PYG{p}{:}
  \PYG{n}{alpha}\PYG{p}{:} \PYG{l+m+mf}{0.25}                           \PYG{c+c1}{\PYGZsh{} mixing parameter for hybrid calculation. If \PYGZdq{}Auto\PYGZdq{} is set, the mixing parameter is set to be one of permittivity and PBE0 calualation is performed.}
  \PYG{n}{cutoff\PYGZus{}df\PYGZus{}dvb}\PYG{p}{:} \PYG{l+m+mf}{0.3}                    \PYG{c+c1}{\PYGZsh{} DF/DVB used to classify semiconductor candidates. (See paper)}
  \PYG{n}{band\PYGZus{}structure\PYGZus{}correction}\PYG{p}{:} \PYG{k+kc}{True}       \PYG{c+c1}{\PYGZsh{} On/Off for the band structure correction}

\PYG{n}{effective\PYGZus{}mass}\PYG{p}{:}
  \PYG{n}{carrier\PYGZus{}type}\PYG{p}{:}                         \PYG{c+c1}{\PYGZsh{} carrier type of effective mass to be estimated}
    \PYG{o}{\PYGZhy{}} \PYG{n}{hole}
    \PYG{o}{\PYGZhy{}} \PYG{n}{electron}
\end{sphinxVerbatim}
\end{quote}


\section{Configuration}
\label{\detokenize{Input/Configuration:configuration}}\label{\detokenize{Input/Configuration::doc}}
AMP$^{\text{2}}$ uses YAML style configuration file. All configurations for AMP$^{\text{2}}$ can
be controlled in “config.yaml”. The default setting parameters are provided
in config\_def.yaml in source directory.

The commands for configuration are listed below.


\subsection{Directory}
\label{\detokenize{Input/Configuration:directory}}
All tags in directory define the path of directories used in AMP$^{\text{2}}$.
If there is no directory in the path for Output, Done and ERROR, AMP$^{\text{2}}$ makes new directories.
\begin{itemize}
\item {} \begin{description}
\item[{Submit:}] \leavevmode
submit tag should be set to be the path for target materials. In AMP$^{\text{2}}$, user
can designate a specific material or a bunch of materials as target materials.
To perform the AMP$^{\text{2}}$ for a specific materials, submit path is set to be the structure
file or the directory for continuous calculation. The valid formats for structure file are
explained in \DUrole{xref,std,std-doc}{/Input\_and\_Output/Input\_files}.
For calculating a bunch of materials, Submit path is set to be the directory where the valid strcture
format files and directories for continuous calculation are placed.

Usage:

\begin{sphinxVerbatim}[commandchars=\\\{\}]
\PYG{n}{directory}\PYG{p}{:}
  \PYG{n}{submit}\PYG{p}{:} \PYG{p}{[}\PYG{n}{path} \PYG{n}{of} \PYG{n}{structure} \PYG{n}{file}\PYG{p}{]} \PYG{o}{\textbar{}} \PYG{p}{[}\PYG{n}{path} \PYG{n}{of} \PYG{n}{directory}\PYG{p}{]}
\end{sphinxVerbatim}

Default:

\begin{sphinxVerbatim}[commandchars=\\\{\}]
\PYG{n}{directory}\PYG{p}{:}
  \PYG{n}{submit}\PYG{p}{:} \PYG{o}{.}\PYG{o}{/}\PYG{n}{Submit}
\end{sphinxVerbatim}

\end{description}

\item {} \begin{description}
\item[{Output:}] \leavevmode
Output tag defines the path where the material on calculation is located.

Usage:

\begin{sphinxVerbatim}[commandchars=\\\{\}]
\PYG{n}{directory}\PYG{p}{:}
  \PYG{n}{output}\PYG{p}{:} \PYG{p}{[}\PYG{n}{path} \PYG{n}{of} \PYG{n}{directory}\PYG{p}{]}
\end{sphinxVerbatim}

Default:

\begin{sphinxVerbatim}[commandchars=\\\{\}]
\PYG{n}{directory}\PYG{p}{:}
  \PYG{n}{output}\PYG{p}{:} \PYG{o}{.}\PYG{o}{/}\PYG{n}{Output}
\end{sphinxVerbatim}

\end{description}

\item {} \begin{description}
\item[{Done:}] \leavevmode
Done tag defines the path where calculated materials are saved.

Usage:

\begin{sphinxVerbatim}[commandchars=\\\{\}]
\PYG{n}{directory}\PYG{p}{:}
  \PYG{n}{done}\PYG{p}{:} \PYG{p}{[}\PYG{n}{path} \PYG{n}{of} \PYG{n}{directory}\PYG{p}{]}
\end{sphinxVerbatim}

Default:

\begin{sphinxVerbatim}[commandchars=\\\{\}]
\PYG{n}{directory}\PYG{p}{:}
  \PYG{n}{done}\PYG{p}{:} \PYG{o}{.}\PYG{o}{/}\PYG{n}{Done}
\end{sphinxVerbatim}

\end{description}

\item {} \begin{description}
\item[{Error:}] \leavevmode
Output tag defines the path saving the materials in which calculation error broke out.

Usage:

\begin{sphinxVerbatim}[commandchars=\\\{\}]
\PYG{n}{directory}\PYG{p}{:}
  \PYG{n}{error}\PYG{p}{:} \PYG{p}{[}\PYG{n}{path} \PYG{n}{of} \PYG{n}{directory}\PYG{p}{]}
\end{sphinxVerbatim}

Default:

\begin{sphinxVerbatim}[commandchars=\\\{\}]
\PYG{n}{directory}\PYG{p}{:}
  \PYG{n}{error}\PYG{p}{:} \PYG{o}{.}\PYG{o}{/}\PYG{n}{ERROR}
\end{sphinxVerbatim}

\end{description}

\item {} \begin{description}
\item[{src\_path:}] \leavevmode
src\_path tag should be set to be the directory for AMP$^{\text{2}}$ source codes.

Usage:

\begin{sphinxVerbatim}[commandchars=\\\{\}]
\PYG{n}{directory}\PYG{p}{:}
  \PYG{n}{src\PYGZus{}path}\PYG{p}{:} \PYG{p}{[}\PYG{n}{path} \PYG{n}{of} \PYG{n}{directory}\PYG{p}{]}
\end{sphinxVerbatim}

Default:

\begin{sphinxVerbatim}[commandchars=\\\{\}]
\PYG{n}{directory}\PYG{p}{:}
  \PYG{n}{src\PYGZus{}path}\PYG{p}{:} \PYG{o}{.}\PYG{o}{/}\PYG{n}{src}
\end{sphinxVerbatim}

\end{description}

\item {} \begin{description}
\item[{pot\_path\_GGA (pot\_path\_LDA):}] \leavevmode
pot\_path\_GGA (pot\_path\_LDA) should be set to be the directory for pseudopotential provided by VASP.

Usage:

\begin{sphinxVerbatim}[commandchars=\\\{\}]
\PYG{n}{directory}\PYG{p}{:}
  \PYG{n}{pot\PYGZus{}path\PYGZus{}GGA}\PYG{p}{:} \PYG{p}{[}\PYG{n}{path} \PYG{n}{of} \PYG{n}{directory}\PYG{p}{]}
  \PYG{n}{pot\PYGZus{}path\PYGZus{}LDA}\PYG{p}{:} \PYG{p}{[}\PYG{n}{path} \PYG{n}{of} \PYG{n}{directory}\PYG{p}{]}
\end{sphinxVerbatim}

Default:

\begin{sphinxVerbatim}[commandchars=\\\{\}]
\PYG{n}{directory}\PYG{p}{:}
  \PYG{n}{pot\PYGZus{}path\PYGZus{}GGA}\PYG{p}{:} \PYG{o}{.}\PYG{o}{/}\PYG{n}{pot}\PYG{o}{/}\PYG{n}{PBE}
  \PYG{n}{pot\PYGZus{}path\PYGZus{}LDA}\PYG{p}{:} \PYG{o}{.}\PYG{o}{/}\PYG{n}{pot}\PYG{o}{/}\PYG{n}{LDA}
\end{sphinxVerbatim}

\end{description}

\end{itemize}


\subsection{Program}
\label{\detokenize{Input/Configuration:program}}
The all tags in program determine the path of executable files except mpi\_command.
\begin{itemize}
\item {} \begin{description}
\item[{vasp\_std:}] \leavevmode
vasp\_std tag should be set to be the path for standard version of VASP.

Usage:

\begin{sphinxVerbatim}[commandchars=\\\{\}]
\PYG{n}{Program}\PYG{p}{:}
  \PYG{n}{vasp\PYGZus{}std}\PYG{p}{:} \PYG{p}{[}\PYG{n}{path}\PYG{p}{]}
\end{sphinxVerbatim}

Default:

\begin{sphinxVerbatim}[commandchars=\\\{\}]
\PYG{n}{Program}\PYG{p}{:}
  \PYG{n}{vasp\PYGZus{}std}\PYG{p}{:} \PYG{o}{.}\PYG{o}{/}\PYG{n}{vasp\PYGZus{}std}
\end{sphinxVerbatim}

\end{description}

\item {} \begin{description}
\item[{vasp\_gam:}] \leavevmode
vasp\_gam tag should be set to be the path for gamma only version of VASP.

Usage:

\begin{sphinxVerbatim}[commandchars=\\\{\}]
\PYG{n}{Program}\PYG{p}{:}
  \PYG{n}{vasp\PYGZus{}gam}\PYG{p}{:} \PYG{p}{[}\PYG{n}{path}\PYG{p}{]}
\end{sphinxVerbatim}

Default:

\begin{sphinxVerbatim}[commandchars=\\\{\}]
\PYG{n}{Program}\PYG{p}{:}
  \PYG{n}{vasp\PYGZus{}gam}\PYG{p}{:} \PYG{o}{.}\PYG{o}{/}\PYG{n}{vasp\PYGZus{}gam}
\end{sphinxVerbatim}

\end{description}

\item {} \begin{description}
\item[{vasp\_ncl:}] \leavevmode
vasp\_ncl tag should be set to be the path for non-collinear version of VASP.
Though wrong path is set, most of calculations except spin-orbit coupling calculation can be conducted.

Usage:

\begin{sphinxVerbatim}[commandchars=\\\{\}]
\PYG{n}{Program}\PYG{p}{:}
  \PYG{n}{vasp\PYGZus{}ncl}\PYG{p}{:} \PYG{p}{[}\PYG{n}{path}\PYG{p}{]}
\end{sphinxVerbatim}

Default:

\begin{sphinxVerbatim}[commandchars=\\\{\}]
\PYG{n}{Program}\PYG{p}{:}
  \PYG{n}{vasp\PYGZus{}ncl}\PYG{p}{:} \PYG{o}{.}\PYG{o}{/}\PYG{n}{vasp\PYGZus{}ncl}
\end{sphinxVerbatim}

\end{description}

\item {} \begin{description}
\item[{gnuplot:}] \leavevmode
gnuplot tag should be set to be the path for gnuplot.
Though wrong path is set, most of calculations except drawing images can be conducted.

Usage:

\begin{sphinxVerbatim}[commandchars=\\\{\}]
\PYG{n}{Program}\PYG{p}{:}
  \PYG{n}{gnuplot}\PYG{p}{:} \PYG{p}{[}\PYG{n}{path}\PYG{p}{]}
\end{sphinxVerbatim}

Default:

\begin{sphinxVerbatim}[commandchars=\\\{\}]
\PYG{n}{Program}\PYG{p}{:}
  \PYG{n}{gnuplot}\PYG{p}{:} \PYG{o}{/}\PYG{n}{usr}\PYG{o}{/}\PYG{n}{local}\PYG{o}{/}\PYG{n+nb}{bin}\PYG{o}{/}\PYG{n}{gnuplot}
\end{sphinxVerbatim}

\end{description}

\item {} \begin{description}
\item[{mpi\_command:}] \leavevmode
mpi\_command tag should be set to be the operation command to conduct parallel computing calculation.
The predefined commands are ’mpirun’, ’jsrun’, ’srun’, ’mpiexec’, ’mpiexec.hydra’, ’mpich’.
Except for the predefined commands, the command should include a flag to specify the number of processors
like ’mpirun -np’.

Usage:

\begin{sphinxVerbatim}[commandchars=\\\{\}]
\PYG{n}{Program}\PYG{p}{:}
  \PYG{n}{mpi\PYGZus{}command}\PYG{p}{:} \PYG{p}{[}\PYG{n}{command}\PYG{p}{]}
\end{sphinxVerbatim}

Default:

\begin{sphinxVerbatim}[commandchars=\\\{\}]
\PYG{n}{Program}\PYG{p}{:}
  \PYG{n}{mpi\PYGZus{}command}\PYG{p}{:} \PYG{n}{mpirun}
\end{sphinxVerbatim}

\end{description}

\end{itemize}


\subsection{Calculation}
\label{\detokenize{Input/Configuration:calculation}}
The all tags in calculation determine whether the calculation is performed or not.
\begin{itemize}
\item {} \begin{description}
\item[{magnetic\_ordering:}] \leavevmode
magnetic\_ordering tag determines whether to identify the most stable magnetic spin ordering or not.

Usage:

\begin{sphinxVerbatim}[commandchars=\\\{\}]
\PYG{n}{Calculation}\PYG{p}{:}
  \PYG{n}{magnetic\PYGZus{}ordering}\PYG{p}{:} \PYG{k+kc}{True} \PYG{o}{\textbar{}} \PYG{k+kc}{False}
\end{sphinxVerbatim}

Default:

\begin{sphinxVerbatim}[commandchars=\\\{\}]
\PYG{n}{Calculation}\PYG{p}{:}
  \PYG{n}{magnetic\PYGZus{}ordering}\PYG{p}{:} \PYG{k+kc}{True}
\end{sphinxVerbatim}

\end{description}

\item {} \begin{description}
\item[{band:}] \leavevmode
band tag determines whether to estimate the band gap and to draw band structure or not.

Usage:

\begin{sphinxVerbatim}[commandchars=\\\{\}]
\PYG{n}{Calculation}\PYG{p}{:}
  \PYG{n}{band}\PYG{p}{:} \PYG{k+kc}{True} \PYG{o}{\textbar{}} \PYG{k+kc}{False}
\end{sphinxVerbatim}

Default:

\begin{sphinxVerbatim}[commandchars=\\\{\}]
\PYG{n}{Calculation}\PYG{p}{:}
  \PYG{n}{band}\PYG{p}{:} \PYG{k+kc}{True}
\end{sphinxVerbatim}

\end{description}

\item {} \begin{description}
\item[{density\_of\_states:}] \leavevmode
density\_of\_states tag determines whether to estimate the density of states or not.

Usage:

\begin{sphinxVerbatim}[commandchars=\\\{\}]
\PYG{n}{Calculation}\PYG{p}{:}
  \PYG{n}{density\PYGZus{}of\PYGZus{}states}\PYG{p}{:} \PYG{k+kc}{True} \PYG{o}{\textbar{}} \PYG{k+kc}{False}
\end{sphinxVerbatim}

Default:

\begin{sphinxVerbatim}[commandchars=\\\{\}]
\PYG{n}{Calculation}\PYG{p}{:}
  \PYG{n}{density\PYGZus{}of\PYGZus{}states}\PYG{p}{:} \PYG{k+kc}{True}
\end{sphinxVerbatim}

\end{description}

\item {} \begin{description}
\item[{hse\_oneshot:}] \leavevmode
hse\_oneshot tag determines whether to perform the hybrid calculation or not. This hybrid calculation
is conducted without full band searching and structure optimization. For hybrid calculation band calculation
must be conducted.

Usage:

\begin{sphinxVerbatim}[commandchars=\\\{\}]
\PYG{n}{Calculation}\PYG{p}{:}
  \PYG{n}{hse\PYGZus{}oneshot}\PYG{p}{:} \PYG{k+kc}{True} \PYG{o}{\textbar{}} \PYG{k+kc}{False}
\end{sphinxVerbatim}

Default:

\begin{sphinxVerbatim}[commandchars=\\\{\}]
\PYG{n}{Calculation}\PYG{p}{:}
  \PYG{n}{hse\PYGZus{}oneshot}\PYG{p}{:} \PYG{k+kc}{True}
\end{sphinxVerbatim}

\end{description}

\item {} \begin{description}
\item[{dielectric:}] \leavevmode
dielectric tag determines whether to estimate the dielectric constant or not. Dielectric constant is
unphysical in metallic system. Thus, band structure calculation must be conducted to check whether
it is metal or not.

Usage:

\begin{sphinxVerbatim}[commandchars=\\\{\}]
\PYG{n}{Calculation}\PYG{p}{:}
  \PYG{n}{dielectric}\PYG{p}{:} \PYG{k+kc}{True} \PYG{o}{\textbar{}} \PYG{k+kc}{False}
\end{sphinxVerbatim}

Default:

\begin{sphinxVerbatim}[commandchars=\\\{\}]
\PYG{n}{Calculation}\PYG{p}{:}
  \PYG{n}{dielectric}\PYG{p}{:} \PYG{k+kc}{True}
\end{sphinxVerbatim}

\end{description}

\item {} \begin{description}
\item[{effective\_mass:}] \leavevmode
effective\_mass tag determines whether to estimate the hole (and/or electron) effective mass or not.
For effective mass calculation band calculation must be conducted.

Usage:

\begin{sphinxVerbatim}[commandchars=\\\{\}]
\PYG{n}{Calculation}\PYG{p}{:}
  \PYG{n}{effective\PYGZus{}mass}\PYG{p}{:} \PYG{k+kc}{True} \PYG{o}{\textbar{}} \PYG{k+kc}{False}
\end{sphinxVerbatim}

Default:

\begin{sphinxVerbatim}[commandchars=\\\{\}]
\PYG{n}{Calculation}\PYG{p}{:}
  \PYG{n}{effective\PYGZus{}mass}\PYG{p}{:} \PYG{k+kc}{True}
\end{sphinxVerbatim}

\end{description}

\item {} \begin{description}
\item[{potential\_type}] \leavevmode
potential\_type tag determines the functional scheme (LDA or GGA) for convergence test. Only one of them should be chosen.

Usage:

\begin{sphinxVerbatim}[commandchars=\\\{\}]
\PYG{n}{Calculation}\PYG{p}{:}
  \PYG{n}{potential\PYGZus{}type}\PYG{p}{:} \PYG{n}{GGA} \PYG{o}{\textbar{}} \PYG{n}{LDA}
\end{sphinxVerbatim}

Default:

\begin{sphinxVerbatim}[commandchars=\\\{\}]
\PYG{n}{Calculation}\PYG{p}{:}
  \PYG{n}{potential\PYGZus{}type}\PYG{p}{:} \PYG{n}{GGA}
\end{sphinxVerbatim}

\end{description}

\end{itemize}


\subsection{Vasp\_parallel}
\label{\detokenize{Input/Configuration:vasp-parallel}}
npar and kpar tags are used to enhance the efficiency of parallel computing calculation of VASP.
\begin{itemize}
\item {} \begin{description}
\item[{npar:}] \leavevmode
napr tag determines the number of bands that are treated in parallel. It is same to NPAR tag in VASP.

Usage:

\begin{sphinxVerbatim}[commandchars=\\\{\}]
\PYG{n}{vasp\PYGZus{}parallel}\PYG{p}{:}
  \PYG{n}{npar}\PYG{p}{:} \PYG{p}{[}\PYG{n}{integer}\PYG{p}{]}
\end{sphinxVerbatim}

Default:

\begin{sphinxVerbatim}[commandchars=\\\{\}]
\PYG{n}{vasp\PYGZus{}parallel}\PYG{p}{:}
  \PYG{n}{npar}\PYG{p}{:} \PYG{l+m+mi}{2}
\end{sphinxVerbatim}

\end{description}

\item {} \begin{description}
\item[{kpar:}] \leavevmode
kpar tag determines the number of kpoints that are treated in parallel. It is same to NPAR tag in VASP.

Usage:

\begin{sphinxVerbatim}[commandchars=\\\{\}]
\PYG{n}{vasp\PYGZus{}parallel}\PYG{p}{:}
  \PYG{n}{kpar}\PYG{p}{:} \PYG{p}{[}\PYG{n}{integer}\PYG{p}{]}
\end{sphinxVerbatim}

Default:

\begin{sphinxVerbatim}[commandchars=\\\{\}]
\PYG{n}{vasp\PYGZus{}parallel}\PYG{p}{:}
  \PYG{n}{kpar}\PYG{p}{:} \PYG{l+m+mi}{2}
\end{sphinxVerbatim}

\end{description}

\end{itemize}


\subsection{cif2vasp}
\label{\detokenize{Input/Configuration:cif2vasp}}
In AMP$^{\text{2}}$, input files for VASP calculation are automatically generated from structure files.
These parameters can control the initial input files for VASP.
\begin{itemize}
\item {} \begin{description}
\item[{pot\_name:}] \leavevmode
pot\_name tag determines the pseudopotential potential for element. By default, the potential file (POTCAR) is built using
the preset pseudopotential. (Preset pseudopotential: \DUrole{xref,std,std-doc}{/Input\_and\_Output/Configuration/potential})

Usage:

\begin{sphinxVerbatim}[commandchars=\\\{\}]
\PYG{n}{cif2vasp}\PYG{p}{:}
  \PYG{n}{pot\PYGZus{}name}\PYG{p}{:}
    \PYG{n}{GGA}\PYG{p}{:}
      \PYG{p}{[}\PYG{n}{element} \PYG{n}{name}\PYG{p}{]}\PYG{p}{:} \PYG{p}{[}\PYG{n+nb}{type} \PYG{n}{of} \PYG{n}{pseudopotential}\PYG{p}{]}
    \PYG{n}{LDA}\PYG{p}{:}
      \PYG{p}{[}\PYG{n}{element} \PYG{n}{name}\PYG{p}{]}\PYG{p}{:} \PYG{p}{[}\PYG{n+nb}{type} \PYG{n}{of} \PYG{n}{pseudopotential}\PYG{p}{]}
\end{sphinxVerbatim}

\end{description}

\item {} \begin{description}
\item[{soc\_target:}] \leavevmode
soc\_target tag determines the elements to carry out spin-orbit coupling calculation. In AMP$^{\text{2}}$, spin-orbit coupling calculation
is performed only for band structure and density of states.

Usage:

\begin{sphinxVerbatim}[commandchars=\\\{\}]
\PYG{n}{cif2vasp}\PYG{p}{:}
  \PYG{n}{soc\PYGZus{}target}\PYG{p}{:}
    \PYG{o}{\PYGZhy{}} \PYG{p}{[}\PYG{n}{element} \PYG{n}{name}\PYG{p}{]}
    \PYG{o}{\PYGZhy{}} \PYG{n}{Bi}
\end{sphinxVerbatim}

Default:

\begin{sphinxVerbatim}[commandchars=\\\{\}]
\PYG{n}{cif2vasp}\PYG{p}{:}
  \PYG{n}{soc\PYGZus{}target}\PYG{p}{:}
\end{sphinxVerbatim}

\end{description}

\item {} \begin{description}
\item[{u\_value:}] \leavevmode
u\_value tag controls \(U\) values for PBE + Hubbard \(U\) method. By default, AMP$^{\text{2}}$ imposes \(U\) parameters for 3d
transition metal. If all tag is used instead of element name, every \(U\) value is set to be the target value.

Usage:

\begin{sphinxVerbatim}[commandchars=\\\{\}]
\PYG{n}{cif2vasp}\PYG{p}{:}
  \PYG{n}{u\PYGZus{}value}\PYG{p}{:}
    \PYG{o}{\PYGZhy{}} \PYG{p}{[}\PYG{n}{element} \PYG{n}{name}\PYG{p}{]}\PYG{p}{:} \PYG{n}{real}
\end{sphinxVerbatim}

Default:

\begin{sphinxVerbatim}[commandchars=\\\{\}]
\PYG{n}{cif2vasp}\PYG{p}{:}
  \PYG{n}{u\PYGZus{}value}\PYG{p}{:}
    \PYG{n}{V}\PYG{p}{:} \PYG{l+m+mf}{3.1}
    \PYG{n}{Cr}\PYG{p}{:} \PYG{l+m+mf}{3.5}
    \PYG{n}{Mn}\PYG{p}{:} \PYG{l+m+mi}{4}
    \PYG{n}{Fe}\PYG{p}{:} \PYG{l+m+mi}{4}
    \PYG{n}{Co}\PYG{p}{:} \PYG{l+m+mf}{3.3}
    \PYG{n}{Ni}\PYG{p}{:} \PYG{l+m+mf}{6.4}
    \PYG{n}{Cu}\PYG{p}{:} \PYG{l+m+mi}{4}
    \PYG{n}{Zn}\PYG{p}{:} \PYG{l+m+mf}{7.5}
\end{sphinxVerbatim}

\end{description}

\end{itemize}


\subsection{Hybrid\_oneshot}
\label{\detokenize{Input/Configuration:hybrid-oneshot}}
Conventional density functional theory calculation like LDA and PBE underestimates band gap and somtimes it gives
wrong results for small gap materials such as Ge and InAs. Thus, AMP$^{\text{2}}$ performs hybrid calculation for accurate band gap.
In the previous study, it is shown that accurate band gap can be obtained using extremum points (valence band maximum and
conduction band minimum) and optimized structure in PBE scheme. Since hybrid calculation demands high computational cost,
this approach is imposed in AMP$^{\text{2}}$.

For the small gap materials with metallic band structure in PBE functionals, DOS (density of states) based correction scheme
is applied in AMP$^{\text{2}}$. (See \DUrole{xref,std,std-doc}{/Input\_and\_Output/Configuration/small\_gap\_correction})

Finally, AMP$^{\text{2}}$ provides a method to select mixing parameter using permittivity since there is an inverse correlation between
mixing parameter and permittivity.
\begin{itemize}
\item {} \begin{description}
\item[{alpha:}] \leavevmode
alpha tag determines a mixing parameter for hybrid calculation. As we mentioned above,
mixing parameter in PBE0 has a inverse correlation with permittivity. If alpha: auto is used,
the mixing parameter is determined as one of permittivity.

Usage:

\begin{sphinxVerbatim}[commandchars=\\\{\}]
\PYG{n}{hybrid\PYGZus{}oneshot}\PYG{p}{:}
  \PYG{n}{alpha}\PYG{p}{:} \PYG{p}{[}\PYG{n}{real}\PYG{p}{]} \PYG{o}{\textbar{}} \PYG{n}{Auto}
\end{sphinxVerbatim}

Default:

\begin{sphinxVerbatim}[commandchars=\\\{\}]
\PYG{n}{hybrid\PYGZus{}oneshot}\PYG{p}{:}
  \PYG{n}{alpha}\PYG{p}{:} \PYG{l+m+mf}{0.25}
\end{sphinxVerbatim}

\end{description}

\item {} \begin{description}
\item[{cutoff\_df\_dvb:}] \leavevmode
cutoff\_df\_dvb tag controls \(D_{\textrm{VB}}/D_{\textrm{F}}\) used to classify semiconductor candidates.

Usage:

\begin{sphinxVerbatim}[commandchars=\\\{\}]
\PYG{n}{hybrid\PYGZus{}oneshot}\PYG{p}{:}
  \PYG{n}{cutoff\PYGZus{}df\PYGZus{}dvb}\PYG{p}{:} \PYG{p}{[}\PYG{n}{real}\PYG{p}{]}
\end{sphinxVerbatim}

Default:

\begin{sphinxVerbatim}[commandchars=\\\{\}]
\PYG{n}{hybrid\PYGZus{}oneshot}\PYG{p}{:}
  \PYG{n}{cutoff\PYGZus{}df\PYGZus{}dvb}\PYG{p}{:} \PYG{l+m+mf}{0.3}
\end{sphinxVerbatim}

\end{description}

\item {} \begin{description}
\item[{band\_structure\_correction:}] \leavevmode
band\_structure\_correction determines whether to conduct scissor-correction for band structure or not.

Usage:

\begin{sphinxVerbatim}[commandchars=\\\{\}]
\PYG{n}{hybrid\PYGZus{}oneshot}\PYG{p}{:}
  \PYG{n}{band\PYGZus{}structure\PYGZus{}correction}\PYG{p}{:} \PYG{k+kc}{True} \PYG{o}{\textbar{}} \PYG{k+kc}{False}
\end{sphinxVerbatim}

Default:

\begin{sphinxVerbatim}[commandchars=\\\{\}]
\PYG{n}{hybrid\PYGZus{}oneshot}\PYG{p}{:}
  \PYG{n}{band\PYGZus{}structure\PYGZus{}correction}\PYG{p}{:} \PYG{k+kc}{True}
\end{sphinxVerbatim}

\end{description}

\end{itemize}


\subsection{Effective\_mass}
\label{\detokenize{Input/Configuration:effective-mass}}
In AMP$^{\text{2}}$, effective mass tensor is estimated using semiclassical transport theory.
The details are explained in the paper.
\begin{itemize}
\item {} \begin{description}
\item[{carrier\_type:}] \leavevmode
carrier\_type tag determines the type of carrier (hole or electron) to be estimated.

Usage:

\begin{sphinxVerbatim}[commandchars=\\\{\}]
\PYG{n}{effective\PYGZus{}mass}\PYG{p}{:}
  \PYG{n}{carrier\PYGZus{}type}\PYG{p}{:}
    \PYG{o}{\PYGZhy{}} \PYG{n}{hole} \PYG{o}{\textbar{}} \PYG{n}{electron}
\end{sphinxVerbatim}

Default:

\begin{sphinxVerbatim}[commandchars=\\\{\}]
\PYG{n}{effective\PYGZus{}mass}\PYG{p}{:}
  \PYG{n}{carrier\PYGZus{}type}\PYG{p}{:}
    \PYG{o}{\PYGZhy{}} \PYG{n}{hole}
    \PYG{o}{\PYGZhy{}} \PYG{n}{electron}
\end{sphinxVerbatim}

\end{description}

\item {} \begin{description}
\item[{temperature\_for\_fermi:}] \leavevmode
It controls the temperature to estimate the hole or electron distribution
based on the Fermi-Dirac function.

Usage:

\begin{sphinxVerbatim}[commandchars=\\\{\}]
\PYG{n}{effective\PYGZus{}mass}\PYG{p}{:}
  \PYG{n}{temperature\PYGZus{}for\PYGZus{}fermi}\PYG{p}{:} \PYG{p}{[}\PYG{n}{real}\PYG{p}{]}
\end{sphinxVerbatim}

Default:

\begin{sphinxVerbatim}[commandchars=\\\{\}]
\PYG{n}{effective\PYGZus{}mass}\PYG{p}{:}
  \PYG{n}{temperature\PYGZus{}for\PYGZus{}fermi}\PYG{p}{:} \PYG{l+m+mi}{300}
\end{sphinxVerbatim}

\end{description}

\end{itemize}


\section{Advanced configuration}
\label{\detokenize{Input/Advanced_configuration:advanced-configuration}}\label{\detokenize{Input/Advanced_configuration::doc}}
For advanced users, AMP$^{\text{2}}$ provides some additional configuration parameters written in
the default configuration file (‘/src/cpnfig\_def.yaml’).

config\_def.yaml:

\begin{sphinxVerbatim}[commandchars=\\\{\}]
\PYG{n}{directory}\PYG{p}{:}
  \PYG{n}{submit}\PYG{p}{:} \PYG{o}{.}\PYG{o}{/}\PYG{n}{Submit}                      \PYG{c+c1}{\PYGZsh{} the path of structure file or the directory containg structure files}
  \PYG{n}{output}\PYG{p}{:} \PYG{o}{.}\PYG{o}{/}\PYG{n}{Output}                      \PYG{c+c1}{\PYGZsh{} the path of the directory where calculation is conducted}
  \PYG{n}{done}\PYG{p}{:} \PYG{o}{.}\PYG{o}{/}\PYG{n}{Done}                          \PYG{c+c1}{\PYGZsh{} the path of the directory where results are saved}
  \PYG{n}{error}\PYG{p}{:} \PYG{o}{.}\PYG{o}{/}\PYG{n}{ERROR}                        \PYG{c+c1}{\PYGZsh{} the path of the directory where the materials with error are saved}
  \PYG{n}{src\PYGZus{}path}\PYG{p}{:} \PYG{o}{.}\PYG{o}{/}\PYG{n}{src}                       \PYG{c+c1}{\PYGZsh{} the path of the directory of AMP2 source codes}
  \PYG{n}{pot\PYGZus{}path\PYGZus{}gga}\PYG{p}{:} \PYG{o}{.}\PYG{o}{/}\PYG{n}{pot}\PYG{o}{/}\PYG{n}{PB}                \PYG{c+c1}{\PYGZsh{} the path of directory for GGA pseudopotential}
  \PYG{n}{pot\PYGZus{}path\PYGZus{}lda}\PYG{p}{:} \PYG{o}{.}\PYG{o}{/}\PYG{n}{pot}\PYG{o}{/}\PYG{n}{LDA}               \PYG{c+c1}{\PYGZsh{} the path of directory for LDA pseudopotential}

\PYG{n}{program}\PYG{p}{:}
  \PYG{n}{vasp\PYGZus{}std}\PYG{p}{:} \PYG{o}{.}\PYG{o}{/}\PYG{n}{vasp\PYGZus{}std}                  \PYG{c+c1}{\PYGZsh{} the path of standard version of VASP}
  \PYG{n}{vasp\PYGZus{}gam}\PYG{p}{:} \PYG{o}{.}\PYG{o}{/}\PYG{n}{vasp\PYGZus{}gam}                  \PYG{c+c1}{\PYGZsh{} the path of gamma\PYGZhy{}only version of VASP}
  \PYG{n}{vasp\PYGZus{}ncl}\PYG{p}{:} \PYG{o}{.}\PYG{o}{/}\PYG{n}{vasp\PYGZus{}ncl}                  \PYG{c+c1}{\PYGZsh{} the path of noncollinear version of VASP}
  \PYG{n}{gnuplot}\PYG{p}{:} \PYG{o}{/}\PYG{n}{usr}\PYG{o}{/}\PYG{n}{local}\PYG{o}{/}\PYG{n+nb}{bin}\PYG{o}{/}\PYG{n}{gnuplot}       \PYG{c+c1}{\PYGZsh{} the path of executable file for gnuplot}
  \PYG{n}{mpi\PYGZus{}command}\PYG{p}{:} \PYG{n}{mpirun}                   \PYG{c+c1}{\PYGZsh{} mpi command (ex. mpirun, mpiexec, ...)}

\PYG{n}{vasp\PYGZus{}parallel}\PYG{p}{:}
  \PYG{n}{npar}\PYG{p}{:} \PYG{l+m+mi}{2}                               \PYG{c+c1}{\PYGZsh{} the number of bands that are treated in parallel. It is same to NPAR tag in VASP.}
  \PYG{n}{kpar}\PYG{p}{:} \PYG{l+m+mi}{2}                               \PYG{c+c1}{\PYGZsh{} the number of kpoints that are treated in parallel. It is same to NPAR tag in VASP.}

\PYG{n}{calculation}\PYG{p}{:}
  \PYG{n}{kp\PYGZus{}test}\PYG{p}{:} \PYG{n}{T}                            \PYG{c+c1}{\PYGZsh{} On/Off for convergence test of k\PYGZhy{}points}
  \PYG{n}{encut\PYGZus{}test}\PYG{p}{:} \PYG{n}{T}                         \PYG{c+c1}{\PYGZsh{} On/Off for convergence test of cutoff energy}
  \PYG{n}{relaxation}\PYG{p}{:} \PYG{n}{T}                         \PYG{c+c1}{\PYGZsh{} On/Off for structure optimization}
  \PYG{n}{magnetic\PYGZus{}ordering}\PYG{p}{:} \PYG{n}{T}                  \PYG{c+c1}{\PYGZsh{} On/Off for calculation to identify most stable magnetic spin ordering}
  \PYG{n}{band}\PYG{p}{:} \PYG{n}{T}                               \PYG{c+c1}{\PYGZsh{} On/Off for the calculation for band structure and band gap}
  \PYG{n}{density\PYGZus{}of\PYGZus{}states}\PYG{p}{:} \PYG{n}{T}                  \PYG{c+c1}{\PYGZsh{} On/Off for the calculation for density of states}
  \PYG{n}{hse\PYGZus{}oneshot}\PYG{p}{:} \PYG{n}{T}                        \PYG{c+c1}{\PYGZsh{} On/Off for the calculation for HSE@PBE}
  \PYG{n}{dielectric}\PYG{p}{:} \PYG{n}{T}                         \PYG{c+c1}{\PYGZsh{} On/Off for the calculation for dielectric constant}
  \PYG{n}{effective\PYGZus{}mass}\PYG{p}{:} \PYG{n}{T}                     \PYG{c+c1}{\PYGZsh{} On/Off for the calculation for effective mass}
  \PYG{n}{potential\PYGZus{}type}\PYG{p}{:} \PYG{n}{GGA}                   \PYG{c+c1}{\PYGZsh{} calculation scheme (LDA or GGA)}

\PYG{n}{cif2vasp}\PYG{p}{:}
  \PYG{n}{pot\PYGZus{}name}\PYG{p}{:}                             \PYG{c+c1}{\PYGZsh{} the pseudopotential potential for element.}
    \PYG{n}{GGA}\PYG{p}{:}                                \PYG{c+c1}{\PYGZsh{} (Ex. GGA:\PYGZbs{}n    Ge:Ge\PYGZus{}d\PYGZbs{}n    Cu:Cu\PYGZus{}pv)}
    \PYG{n}{LDA}\PYG{p}{:}
  \PYG{n}{soc\PYGZus{}target}\PYG{p}{:}                           \PYG{c+c1}{\PYGZsh{} the elements to carry out spin\PYGZhy{}orbit coupling calculation (Ex. soc\PYGZus{}target:\PYGZbs{}n    \PYGZhy{} Bi\PYGZbs{}n    \PYGZhy{} Pb)}
  \PYG{n}{u\PYGZus{}value}\PYG{p}{:}                              \PYG{c+c1}{\PYGZsh{} U values for PBE+U calculation (Ex. u\PYGZus{}value:\PYGZbs{}n    La: 7.5\PYGZbs{}n    Ce: 8.5)}
  \PYG{n}{max\PYGZus{}nelm}\PYG{p}{:} \PYG{l+m+mi}{100}                         \PYG{c+c1}{\PYGZsh{} the maximum number of iteration for structure optimization.}

\PYG{n}{convergence\PYGZus{}test}\PYG{p}{:}
  \PYG{n}{enconv}\PYG{p}{:} \PYG{l+m+mf}{0.01}                          \PYG{c+c1}{\PYGZsh{} convergence condition for energy (eV/atom). Negative value indicates that energy is not used as the condition.}
  \PYG{n}{prconv}\PYG{p}{:} \PYG{l+m+mi}{10}                            \PYG{c+c1}{\PYGZsh{} convergence condition for pressure (bar). Negative value indicates that pressure is not used as the condition.}
  \PYG{n}{foconv}\PYG{p}{:} \PYG{o}{\PYGZhy{}}\PYG{l+m+mi}{1}                            \PYG{c+c1}{\PYGZsh{} convergence condition for force (eV/angst). Negative value indicates that force is not used as the condition.}
  \PYG{n}{initial\PYGZus{}kpl}\PYG{p}{:} \PYG{l+m+mi}{1}                        \PYG{c+c1}{\PYGZsh{} Minimum value for the convergence test of k\PYGZhy{}points. It corresponds to the largest mesh grid in the three directions.}
  \PYG{n}{max\PYGZus{}kpl}\PYG{p}{:} \PYG{l+m+mi}{20}                           \PYG{c+c1}{\PYGZsh{} Maximum value for the convergence test of k\PYGZhy{}points. It corresponds to the largest mesh grid in the three directions.}
  \PYG{n}{enstart}\PYG{p}{:} \PYG{l+m+mi}{200}                          \PYG{c+c1}{\PYGZsh{} Minimum value for the convergence test of cutoff energy}
  \PYG{n}{enstep}\PYG{p}{:} \PYG{l+m+mi}{50}                            \PYG{c+c1}{\PYGZsh{} Interval for the convergence test of cutoff energy}
  \PYG{n}{enmax}\PYG{p}{:} \PYG{l+m+mi}{1000}                           \PYG{c+c1}{\PYGZsh{} Maximum value for the convergence test of cutoff energy}
  \PYG{n}{potential\PYGZus{}type}\PYG{p}{:} \PYG{n}{GGA}                   \PYG{c+c1}{\PYGZsh{} Calculation scheme for convergence test. User have to choose one potential among the GGA, LDA and HSE.}

\PYG{n}{relaxation}\PYG{p}{:}
  \PYG{n}{potential\PYGZus{}type}\PYG{p}{:}                        \PYG{c+c1}{\PYGZsh{} Calculation scheme for structure optimization. User can choose one or more potential among the GGA, LDA and HSE.}
    \PYG{o}{\PYGZhy{}} \PYG{n}{GGA}
  \PYG{n}{max\PYGZus{}iteration}\PYG{p}{:} \PYG{l+m+mi}{10}                     \PYG{c+c1}{\PYGZsh{} The maximum iteration number from previously optimized structure}
  \PYG{n}{converged\PYGZus{}ionic\PYGZus{}step}\PYG{p}{:} \PYG{o}{\PYGZhy{}}\PYG{l+m+mi}{1}              \PYG{c+c1}{\PYGZsh{} The tolerance of steps for iteration. Until the relaxation finishes within the tolerance, we iterate the structure relaxation from previously optimized structure. In negative value, it is neglected.}
  \PYG{n}{length\PYGZus{}tolerance}\PYG{p}{:} \PYG{l+m+mf}{0.002}               \PYG{c+c1}{\PYGZsh{} The tolerance of length (ratio). Until the relaxation finishes within the tolerance, we iterate the structure relaxation from previously optimized structure. In negative value, it is neglected.}
  \PYG{n}{angle\PYGZus{}tolerance}\PYG{p}{:} \PYG{l+m+mf}{0.01}                 \PYG{c+c1}{\PYGZsh{} The tolerance of angle (degrees). Until the relaxation finishes within the tolerance, we iterate the structure relaxation from previously optimized structure. In negative value, it is neglected.}
  \PYG{n}{energy}\PYG{p}{:} \PYG{o}{\PYGZhy{}}\PYG{l+m+mi}{1}                            \PYG{c+c1}{\PYGZsh{} The energy tolerance (eV) to break the loop for structure optimization in VASP. In negative value, it is neglected.}
  \PYG{n}{pressure}\PYG{p}{:} \PYG{l+m+mi}{10}                          \PYG{c+c1}{\PYGZsh{} The pressure tolerance (bar) to break the loop for structure optimization in VASP. In negative value, it is neglected.}
  \PYG{n}{force}\PYG{p}{:} \PYG{l+m+mf}{0.02}                           \PYG{c+c1}{\PYGZsh{} The force tolerance (eV/angst) to break the loop for structure optimization in VASP. In negative value, it is neglected.}

\PYG{n}{band\PYGZus{}calculation}\PYG{p}{:}
  \PYG{n}{kspacing\PYGZus{}for\PYGZus{}band}\PYG{p}{:} \PYG{l+m+mf}{0.01}               \PYG{c+c1}{\PYGZsh{} The distance between adjacent points in the band structure (2pi/ang).}
  \PYG{n}{type\PYGZus{}of\PYGZus{}kpt}\PYG{p}{:} \PYG{n+nb}{all}                      \PYG{c+c1}{\PYGZsh{} Set the lines to calculate the band gap. In the \PYGZsq{}all\PYGZsq{}, AMP2 calculates the eigenvalues along the lines connecting every combination of high symmetric k\PYGZhy{}points. In the \PYGZsq{}band\PYGZsq{}, AMP2 calculates the eigenvalue along the line to draw band structure.}
  \PYG{n}{y\PYGZus{}min}\PYG{p}{:} \PYG{l+m+mi}{3}                              \PYG{c+c1}{\PYGZsh{} The minimum energy range for band structure.}
  \PYG{n}{y\PYGZus{}max}\PYG{p}{:} \PYG{l+m+mi}{2}                              \PYG{c+c1}{\PYGZsh{} The maximum energy range from conduction band minimum for band structure.}
  \PYG{n}{potential\PYGZus{}type}\PYG{p}{:}                       \PYG{c+c1}{\PYGZsh{} Calculation scheme for band structure. User can choose one or more potential among the GGA, LDA and HSE.}
    \PYG{o}{\PYGZhy{}} \PYG{n}{GGA}

\PYG{n}{density\PYGZus{}of\PYGZus{}states}\PYG{p}{:}
  \PYG{n}{kp\PYGZus{}multiplier}\PYG{p}{:} \PYG{n+nb}{all}                    \PYG{c+c1}{\PYGZsh{} Multiplier for k\PYGZhy{}points for smooth figure.}
  \PYG{n}{y\PYGZus{}min}\PYG{p}{:} \PYG{l+m+mi}{3}                              \PYG{c+c1}{\PYGZsh{} The minimum energy range for density of states.}
  \PYG{n}{y\PYGZus{}max}\PYG{p}{:} \PYG{l+m+mi}{2}                              \PYG{c+c1}{\PYGZsh{} The maximum energy range from conduction band minimum for density of states.}
  \PYG{n}{potential\PYGZus{}type}\PYG{p}{:}                       \PYG{c+c1}{\PYGZsh{} Calculation scheme for density of states. User can choose one or more potential among the GGA, LDA and HSE.}
    \PYG{o}{\PYGZhy{}} \PYG{n}{GGA}

\PYG{n}{hybrid\PYGZus{}oneshot}\PYG{p}{:}
  \PYG{n}{alpha}\PYG{p}{:} \PYG{l+m+mf}{0.25}                           \PYG{c+c1}{\PYGZsh{} Mixing parameter for hybrid calculation. If \PYGZsq{}Auto\PYGZsq{} is set, the mixing parameter is set to be one of permittivity and PBE0 calualation is performed.}
  \PYG{n}{fermi\PYGZus{}width}\PYG{p}{:} \PYG{l+m+mf}{0.3}                      \PYG{c+c1}{\PYGZsh{} The energy range for DF}
  \PYG{n}{vb\PYGZus{}dos\PYGZus{}min}\PYG{p}{:} \PYG{l+m+mi}{1}                         \PYG{c+c1}{\PYGZsh{} The energy range for DVB}
  \PYG{n}{vb\PYGZus{}dos\PYGZus{}max}\PYG{p}{:} \PYG{l+m+mi}{3}                         \PYG{c+c1}{\PYGZsh{} The energy range for DVB}
  \PYG{n}{cutoff\PYGZus{}df\PYGZus{}dvb}\PYG{p}{:} \PYG{l+m+mf}{0.3}                    \PYG{c+c1}{\PYGZsh{} DF/DVB used to classify semiconductor candidates. (See paper)}
  \PYG{n}{band\PYGZus{}structure\PYGZus{}correction}\PYG{p}{:} \PYG{k+kc}{True}       \PYG{c+c1}{\PYGZsh{} On/Off for the band structure correction}
  \PYG{n}{potential\PYGZus{}type}\PYG{p}{:}                       \PYG{c+c1}{\PYGZsh{} The potential used for lattice parameter optimization and for identifying the points at VBM and VBM. If one variable is inserted, AMP2 uses the lattice parameter and the points of VBM and CBM with that potential. If two variables are inserted, AMP2 uses the lattice parameter with above potential and the points of VBM and CBM with below potential. (Ex. potential\PYGZus{}type:\PYGZbs{}n    \PYGZhy{} \PYGZhy{} HSE\PYGZbs{}n      \PYGZhy{} GGA)}
    \PYG{o}{\PYGZhy{}} \PYG{n}{GGA}

\PYG{n}{dielectric}\PYG{p}{:}
  \PYG{n}{kp\PYGZus{}multiplier}\PYG{p}{:} \PYG{n+nb}{all}                    \PYG{c+c1}{\PYGZsh{} Multiplier for k\PYGZhy{}points for dielectric constant.}
  \PYG{n}{potential\PYGZus{}type}\PYG{p}{:}                       \PYG{c+c1}{\PYGZsh{} Calculation scheme for dielectric constant. User can choose one or more potential among the GGA and LDA}
    \PYG{o}{\PYGZhy{}} \PYG{n}{GGA}

\PYG{n}{effective\PYGZus{}mass}\PYG{p}{:}
  \PYG{n}{carrier\PYGZus{}type}\PYG{p}{:}                         \PYG{c+c1}{\PYGZsh{} carrier type of effective mass to be estimated}
    \PYG{o}{\PYGZhy{}} \PYG{n}{hole}
    \PYG{o}{\PYGZhy{}} \PYG{n}{electron}
  \PYG{n}{temperature\PYGZus{}for\PYGZus{}fermi}\PYG{p}{:} \PYG{l+m+mi}{300}            \PYG{c+c1}{\PYGZsh{} The temperature to estimate the Fermi distribution}
  \PYG{n}{fermi\PYGZus{}for\PYGZus{}cutoff}\PYG{p}{:} \PYG{l+m+mf}{0.99}                \PYG{c+c1}{\PYGZsh{} Boundary condition for valid Fermi distribution (1\PYGZhy{}f)}
\end{sphinxVerbatim}


\subsection{To get more accurate band gap}
\label{\detokenize{Input/Advanced_configuration:to-get-more-accurate-band-gap}}
We suggest two approaches to get more accurate band gap.
\begin{itemize}
\item {} 
Band calculation with hybrid functional

In the basic version, the band calculation is performed using PBE scheme.
However, users can add the tags below to use hybrid functional for structure
optimization and band calculation.

\begin{sphinxVerbatim}[commandchars=\\\{\}]
\PYG{n}{relaxation}\PYG{p}{:}
  \PYG{n}{potential\PYGZus{}type}\PYG{p}{:}
    \PYG{o}{\PYGZhy{}} \PYG{n}{HSE}
\PYG{n}{band\PYGZus{}calculation}\PYG{p}{:}
  \PYG{n}{potential\PYGZus{}type}\PYG{p}{:}
    \PYG{o}{\PYGZhy{}} \PYG{n}{HSE}
\end{sphinxVerbatim}

\item {} 
Using \sphinxhref{mailto:HSE@PBE}{HSE@PBE} scheme with hybrid structure

Second approach is still using \sphinxhref{mailto:HSE@PBE}{HSE@PBE} method but the optimized structure is
calculated using hybrid functional. Since the band calculation with hybrid functional
is too expensive, the k-points corresponding to the VBM and CBM are determined by using
GGA method. For this calculation, users can use the commands below. Here, if potential\_type
in hybrid\_oneshot is the main category, the method tags (HSE and GGA) are child subcategory
not parent subcategory. Please be careful.

\begin{sphinxVerbatim}[commandchars=\\\{\}]
\PYG{n}{relaxation}\PYG{p}{:}
  \PYG{n}{potential\PYGZus{}type}\PYG{p}{:}
    \PYG{o}{\PYGZhy{}} \PYG{n}{GGA}
    \PYG{o}{\PYGZhy{}} \PYG{n}{HSE}
\PYG{n}{hybrid\PYGZus{}oneshot}\PYG{p}{:}
  \PYG{n}{potential\PYGZus{}type}\PYG{p}{:}
    \PYG{o}{\PYGZhy{}} \PYG{o}{\PYGZhy{}} \PYG{n}{HSE}
      \PYG{o}{\PYGZhy{}} \PYG{n}{GGA}
\end{sphinxVerbatim}

\end{itemize}


\subsection{Organic crystal}
\label{\detokenize{Input/Advanced_configuration:organic-crystal}}
Organic crystals usually have lower Young’s modulus than inorganic materials.
Thus, the error in the structural parameters can be substantial and they require
high precision for calculation. The tags below can control the precision of calculation.

\begin{sphinxVerbatim}[commandchars=\\\{\}]
\PYG{n}{cif2vasp}\PYG{p}{:}
  \PYG{n}{INCAR}\PYG{p}{:}
    \PYG{n}{EDIFF}\PYG{p}{:} \PYG{l+m+mf}{1e\PYGZhy{}08}

\PYG{n}{convergence\PYGZus{}test}\PYG{p}{:}
  \PYG{n}{enconv}\PYG{p}{:} \PYG{l+m+mf}{0.001}
  \PYG{n}{prconv}\PYG{p}{:} \PYG{l+m+mi}{1}

\PYG{n}{relaxation}\PYG{p}{:}
  \PYG{n}{pressure}\PYG{p}{:} \PYG{l+m+mi}{1}
  \PYG{n}{force}\PYG{p}{:} \PYG{l+m+mf}{0.002}
\end{sphinxVerbatim}


\chapter{Output}
\label{\detokenize{Output/Output:output}}\label{\detokenize{Output/Output::doc}}

\section{Output files}
\label{\detokenize{Output/Output:output-files}}
AMP$^{\text{2}}$ makes directory for each configuration file as its name (from name.cif or POSCAR\_name).
When the calculation is on progress, the directory is
placed in output path in the configuration. If calculation is well finished, the calculation directory
is moved to done path. If any error breaks out, it is move to error path.

Each directory includes several sub-directory as follow;


\subsection{INPUT0}
\label{\detokenize{Output/Output:input0}}\begin{quote}

Directory for input files for VASP calculation.
\begin{itemize}
\item {} \begin{description}
\item[{POSCAR\_rlx\_POT:}] \leavevmode
Optimized structure file with POT functional.

\end{description}

\item {} \begin{description}
\item[{KPOINTS:}] \leavevmode
Converged k-points file

\end{description}

\item {} \begin{description}
\item[{INCAR:}] \leavevmode
VASP input file with converged cutoff energy and ground-state magnetic ordering

\end{description}

\end{itemize}
\end{quote}


\subsection{kptest}
\label{\detokenize{Output/Output:kptest}}\begin{quote}

Directory for k-point convergence test.
\begin{itemize}
\item {} \begin{description}
\item[{kpoint.log:}] \leavevmode
Calculation log for k-points convergence test

\end{description}

\end{itemize}
\end{quote}


\subsection{encut}
\label{\detokenize{Output/Output:encut}}\begin{quote}

Directory for cutoff energy convergence test.
\begin{itemize}
\item {} \begin{description}
\item[{cutoff.log:}] \leavevmode
Calculation log for cutoff energy convergence test

\end{description}

\end{itemize}
\end{quote}


\subsection{relax\_POT (POT = GGA or LDA)}
\label{\detokenize{Output/Output:relax-pot-pot-gga-or-lda}}\begin{quote}

Directory for structure relaxation.
\end{quote}


\subsection{magnetic\_ordering}
\label{\detokenize{Output/Output:magnetic-ordering}}\begin{quote}

Directory for identifying magnetic spin ordering.
\end{quote}


\subsection{band\_POT (POT = GGA or LDA)}
\label{\detokenize{Output/Output:band-pot-pot-gga-or-lda}}\begin{quote}

Directory for band structure and band gap calculation.
\end{quote}


\subsection{dos\_POT (POT = GGA or LDA)}
\label{\detokenize{Output/Output:dos-pot-pot-gga-or-lda}}\begin{quote}

Directory for density of states calculation.
\end{quote}


\subsection{dielectric\_POT (POT = GGA or LDA)}
\label{\detokenize{Output/Output:dielectric-pot-pot-gga-or-lda}}\begin{quote}

Directory for dielectric constant calculation.
\end{quote}


\subsection{hybrid\_POT1\_POT2 (POT = GGA or LDA)}
\label{\detokenize{Output/Output:hybrid-pot1-pot2-pot-gga-or-lda}}\begin{quote}

Directory for band gap calculation with hybrid oneshot scheme.
\end{quote}


\subsection{effm\_POT (POT = GGA or LDA)}
\label{\detokenize{Output/Output:effm-pot-pot-gga-or-lda}}\begin{quote}

Directory for effective mass calculation.
\end{quote}


\subsection{Results}
\label{\detokenize{Output/Output:results}}\begin{quote}

Directory for calculation results.
\begin{itemize}
\item {} \begin{description}
\item[{POSCAR\_GGA:}] \leavevmode
Optimized structure

\end{description}

\item {} \begin{description}
\item[{Band\_gap\_GGA.log:}] \leavevmode
Information of band gap

\end{description}

\item {} \begin{description}
\item[{band\_GGA.png (band\_GGA.pdf):}] \leavevmode
Band structure image

\end{description}

\item {} \begin{description}
\item[{band\_corrected.png (band\_corrected.pdf):}] \leavevmode
Corrected band structure image

\end{description}

\item {} \begin{description}
\item[{Band\_gap\_hybrid\_GGA.log:}] \leavevmode
Information of band gap with \sphinxhref{mailto:HSE@PBE}{HSE@PBE} scheme

\end{description}

\item {} \begin{description}
\item[{dos\_GGA.png (dos\_GGA.pdf):}] \leavevmode
Density of states image

\end{description}

\item {} \begin{description}
\item[{dielectric\_GGA.log:}] \leavevmode
Information of dielectric constant

\end{description}

\item {} \begin{description}
\item[{effective\_mass\_hole\_GGA.log:}] \leavevmode
Information of effective mass of hole

\end{description}

\item {} \begin{description}
\item[{effective\_mass\_electron\_GGA.log:}] \leavevmode
Information of effective mass of electron

\end{description}

\item {} \begin{description}
\item[{Properties.json:}] \leavevmode
Summarized material properties

\end{description}

\end{itemize}
\end{quote}


\subsection{INPUT0\_old}
\label{\detokenize{Output/Output:input0-old}}\begin{quote}

Directory for input files for VASP calculation with ferromagnetic ordering.
If more stable magnetic spin ordering is obsevred, this directory is made.
\end{quote}


\subsection{relax\_POT\_old (POT = GGA or LDA)}
\label{\detokenize{Output/Output:relax-pot-old-pot-gga-or-lda}}\begin{quote}

Directory for structure relaxation with ferromagnetic ordering.
If more stable magnetic spin ordering is obsevred, this directory is made.
\end{quote}


\subsection{name\_with\_U}
\label{\detokenize{Output/Output:name-with-u}}\begin{quote}

Directory for AMP$^{\text{2}}$ calculation with DFT+U calculation.
If the material is metallic and DFT+U calculation has been conducted,
all of results move to this directory.
\end{quote}

Additionally, AMP$^{\text{2}}$ provides log file as amp2.log for tracing the calculation.


\chapter{Examples}
\label{\detokenize{Examples/Examples:examples}}\label{\detokenize{Examples/Examples::doc}}

\section{Introduction}
\label{\detokenize{Examples/Examples:introduction}}
AMP$^{\text{2}}$includes several examples (for Si, Ge and NiO) in \sphinxstyleemphasis{AMP2/examples/}.


\section{Execute AMP$^{\text{2}}$}
\label{\detokenize{Examples/Examples:execute-amp2}}
Before running examples, please set the configuration to be suitable for your system.
(See {\hyperref[\detokenize{Installation/Installation::doc}]{\sphinxcrossref{\DUrole{doc}{Installation and execution}}}})
Then, you can execute AMP$^{\text{2}}$ using shell script as following.

\begin{sphinxVerbatim}[commandchars=\\\{\}]
\PYG{n}{sh} \PYG{n}{run}\PYG{o}{.}\PYG{n}{sh}
\end{sphinxVerbatim}


\section{Calculation results}
\label{\detokenize{Examples/Examples:calculation-results}}
When the calculation is finished, Sub-directory is generated in Done path. (Ex. \sphinxstyleemphasis{/Done/Si})
In the \sphinxstyleemphasis{Sub-directory/Results}, you can obtain optimized structure, band gap, band structure, density of states,
dielectric constant and effective mass of hole and electron.


\subsection{Si}
\label{\detokenize{Examples/Examples:si}}
Si is a typical example of semiconductor. Therefore, we calculate all properties supported by AMP$^{\text{2}}$in this example.
\begin{quote}

Optimized structure (/Results/POSCAR\_rlx\_GGA)

\begin{sphinxVerbatim}[commandchars=\\\{\}]
relaxed poscar
1.000000000
    0.0    2.73243086189    2.73243086189
    2.73243086189    \PYGZhy{}0.0    2.73243086189
    2.73243086189    2.73243086189    0.0
    Si
    2
Selective dynamics
Direct
    0.5    0.5    0.5  T  T  T ! Si1
    0.75    0.75    0.75  T  T  T ! Si1
\end{sphinxVerbatim}

Band gap (/Results/band\_gap\_GGA.log)

\begin{sphinxVerbatim}[commandchars=\\\{\}]
\PYG{n}{Band} \PYG{n}{gap}\PYG{p}{:}      \PYG{l+m+mf}{0.612} \PYG{n}{eV} \PYG{p}{(}\PYG{n}{Indirect}\PYG{p}{)}

\PYG{n}{VBM}\PYG{p}{:} \PYG{l+m+mf}{0.0}  \PYG{l+m+mf}{0.0}  \PYG{l+m+mf}{0.0}   \PYG{p}{:}      \PYG{l+m+mf}{5.649} \PYG{n}{eV}
\PYG{n}{CBM}\PYG{p}{:} \PYG{l+m+mf}{0.4166667}  \PYG{l+m+mf}{0.0}  \PYG{l+m+mf}{0.4166667}   \PYG{p}{:}      \PYG{l+m+mf}{6.261} \PYG{n}{eV}

\PYG{n}{nVBM}\PYG{p}{:} \PYG{l+m+mi}{4}  \PYG{n}{spin}\PYG{p}{:} \PYG{l+m+mi}{1}
\PYG{n}{nCBM}\PYG{p}{:} \PYG{l+m+mi}{5}  \PYG{n}{spin}\PYG{p}{:} \PYG{l+m+mi}{1}
\end{sphinxVerbatim}

Band structure (/Results/band\_GGA.png and /Results/band\_GGA.pdf)
\begin{quote}

\noindent\sphinxincludegraphics[width=300\sphinxpxdimen]{{band_GGA}.png}
\end{quote}

Band gap from \sphinxhref{mailto:HSE@PBE}{HSE@PBE} (/Results/band\_hybrid\_GGA.log)

\begin{sphinxVerbatim}[commandchars=\\\{\}]
\PYG{n}{Band} \PYG{n}{gap}\PYG{p}{:}      \PYG{l+m+mf}{1.187} \PYG{n}{eV} \PYG{p}{(}\PYG{n}{Indirect}\PYG{p}{)}

\PYG{n}{VBM}\PYG{p}{:} \PYG{l+m+mf}{0.0}  \PYG{l+m+mf}{0.0}  \PYG{l+m+mf}{0.0}   \PYG{p}{:}      \PYG{l+m+mf}{5.289} \PYG{n}{eV}
\PYG{n}{CBM}\PYG{p}{:} \PYG{l+m+mf}{0.4166667}  \PYG{l+m+mf}{0.0}  \PYG{l+m+mf}{0.4166667}   \PYG{p}{:}      \PYG{l+m+mf}{6.477} \PYG{n}{eV}

\PYG{n}{nVBM}\PYG{p}{:} \PYG{l+m+mi}{4}  \PYG{n}{spin}\PYG{p}{:} \PYG{l+m+mi}{1}
\PYG{n}{nCBM}\PYG{p}{:} \PYG{l+m+mi}{5}  \PYG{n}{spin}\PYG{p}{:} \PYG{l+m+mi}{1}
\end{sphinxVerbatim}

Corrected band structure (/Results/band\_GGA\_corrected.png and /Results/band\_GGA\_corrected.pdf)
\begin{quote}

\noindent\sphinxincludegraphics[width=300\sphinxpxdimen]{{band_GGA_corrected}.png}
\end{quote}

Density of states (/Results/dos\_GGA.log)
\begin{quote}

\noindent\sphinxincludegraphics[width=150\sphinxpxdimen]{{dos_GGA}.png}
\end{quote}

Dielectric constant (/Results/dielectric\_GGA.log)

\begin{sphinxVerbatim}[commandchars=\\\{\}]
\PYG{n}{Dielectric} \PYG{n}{tensor} \PYG{p}{(}\PYG{n}{electronic} \PYG{n}{contribution}\PYG{p}{)}\PYG{p}{:}
    \PYG{l+m+mf}{12.936}       \PYG{l+m+mf}{0.000}      \PYG{o}{\PYGZhy{}}\PYG{l+m+mf}{0.000}
    \PYG{l+m+mf}{0.000}      \PYG{l+m+mf}{12.936}       \PYG{l+m+mf}{0.000}
    \PYG{o}{\PYGZhy{}}\PYG{l+m+mf}{0.000}       \PYG{l+m+mf}{0.000}      \PYG{l+m+mf}{12.936}
\PYG{n}{Dielectric} \PYG{n}{tensor} \PYG{p}{(}\PYG{n}{ionic} \PYG{n}{contribution}\PYG{p}{)}\PYG{p}{:}
    \PYG{l+m+mf}{0.000}       \PYG{l+m+mf}{0.000}       \PYG{l+m+mf}{0.000}
    \PYG{l+m+mf}{0.000}      \PYG{o}{\PYGZhy{}}\PYG{l+m+mf}{0.000}      \PYG{o}{\PYGZhy{}}\PYG{l+m+mf}{0.000}
    \PYG{l+m+mf}{0.000}      \PYG{o}{\PYGZhy{}}\PYG{l+m+mf}{0.000}       \PYG{l+m+mf}{0.000}

\PYG{n}{Dielectric} \PYG{n}{constant} \PYG{n}{diagonalization} \PYG{p}{(}\PYG{n}{electronic}\PYG{p}{)}\PYG{p}{:}     \PYG{l+m+mf}{12.936}     \PYG{l+m+mf}{12.936}     \PYG{l+m+mf}{12.936}
\PYG{n}{Dielectric} \PYG{n}{constant} \PYG{n}{diagonalization} \PYG{p}{(}\PYG{n}{ionic}\PYG{p}{)}\PYG{p}{:}      \PYG{l+m+mf}{0.000}     \PYG{o}{\PYGZhy{}}\PYG{l+m+mf}{0.000}      \PYG{l+m+mf}{0.000}

\PYG{n}{Averaged} \PYG{n}{static} \PYG{n}{dielectric} \PYG{n}{constant}\PYG{p}{:}     \PYG{l+m+mf}{12.936}
\end{sphinxVerbatim}

Effective mass of hole (/effective\_mass\_hole\_GGA.log)

\begin{sphinxVerbatim}[commandchars=\\\{\}]
\PYG{n}{hole}
    \PYG{o}{\PYGZhy{}}\PYG{l+m+mf}{0.266}     \PYG{o}{\PYGZhy{}}\PYG{l+m+mf}{0.000}     \PYG{o}{\PYGZhy{}}\PYG{l+m+mf}{0.000}
    \PYG{o}{\PYGZhy{}}\PYG{l+m+mf}{0.000}     \PYG{o}{\PYGZhy{}}\PYG{l+m+mf}{0.266}     \PYG{o}{\PYGZhy{}}\PYG{l+m+mf}{0.000}
    \PYG{o}{\PYGZhy{}}\PYG{l+m+mf}{0.000}     \PYG{o}{\PYGZhy{}}\PYG{l+m+mf}{0.000}     \PYG{o}{\PYGZhy{}}\PYG{l+m+mf}{0.266}
\PYG{n}{Diagonalized} \PYG{n}{effective} \PYG{n}{mass}\PYG{p}{:}     \PYG{o}{\PYGZhy{}}\PYG{l+m+mf}{0.266}     \PYG{o}{\PYGZhy{}}\PYG{l+m+mf}{0.266}     \PYG{o}{\PYGZhy{}}\PYG{l+m+mf}{0.266}
\end{sphinxVerbatim}

Effective mass of electron (/Results/effective\_mass\_hole\_GGA.log)

\begin{sphinxVerbatim}[commandchars=\\\{\}]
\PYG{n}{electron}
    \PYG{l+m+mf}{0.287}      \PYG{l+m+mf}{0.000}      \PYG{l+m+mf}{0.000}
    \PYG{l+m+mf}{0.000}      \PYG{l+m+mf}{0.287}      \PYG{l+m+mf}{0.000}
    \PYG{l+m+mf}{0.000}      \PYG{l+m+mf}{0.000}      \PYG{l+m+mf}{0.287}
\PYG{n}{Diagonalized} \PYG{n}{effective} \PYG{n}{mass}\PYG{p}{:}      \PYG{l+m+mf}{0.287}      \PYG{l+m+mf}{0.287}      \PYG{l+m+mf}{0.287}
\end{sphinxVerbatim}
\end{quote}


\subsection{Ge}
\label{\detokenize{Examples/Examples:ge}}
Ge is a well-known semiconductor with metallic band structure in conventional DFT schemes like LDA and PBE.
In AMP$^{\text{2}}$, however, we can obtain the reliable band gap and band structure due to the band gap correction
scheme. In this example, we calculate corrected band structure.
\begin{quote}

Band gap (/Results/band\_gap\_GGA.log)

\begin{sphinxVerbatim}[commandchars=\\\{\}]
This system is metallic.
! If it is not hybrid calculation, additional search is required for hybrid calculation.
\end{sphinxVerbatim}

Band structure (/Results/band\_GGA.png and /Results/band\_GGA.pdf)
\begin{quote}

\noindent\sphinxincludegraphics[width=300\sphinxpxdimen]{{band_GGA1}.png}
\end{quote}

Band gap from \sphinxhref{mailto:HSE@PBE}{HSE@PBE} (/Results/band\_hybrid\_GGA.log)

\begin{sphinxVerbatim}[commandchars=\\\{\}]
\PYG{n}{Band} \PYG{n}{gap}\PYG{p}{:}      \PYG{l+m+mf}{0.161} \PYG{n}{eV} \PYG{p}{(}\PYG{n}{Direct}\PYG{p}{)}

\PYG{n}{VBM}\PYG{p}{:} \PYG{l+m+mf}{0.0}  \PYG{l+m+mf}{0.0}  \PYG{l+m+mf}{0.0}   \PYG{p}{:}      \PYG{l+m+mf}{2.875} \PYG{n}{eV}
\PYG{n}{CBM}\PYG{p}{:} \PYG{l+m+mf}{0.0}  \PYG{l+m+mf}{0.0}  \PYG{l+m+mf}{0.0}   \PYG{p}{:}      \PYG{l+m+mf}{3.036} \PYG{n}{eV}

\PYG{n}{nVBM}\PYG{p}{:} \PYG{l+m+mi}{4}  \PYG{n}{spin}\PYG{p}{:} \PYG{l+m+mi}{1}
\PYG{n}{nCBM}\PYG{p}{:} \PYG{l+m+mi}{5}  \PYG{n}{spin}\PYG{p}{:} \PYG{l+m+mi}{1}
\end{sphinxVerbatim}

Corrected band structure (/Results/band\_GGA\_corrected.png and /Results/band\_GGA\_corrected.pdf)
\begin{quote}

\noindent\sphinxincludegraphics[width=300\sphinxpxdimen]{{band_GGA_corrected1}.png}
\end{quote}
\end{quote}


\subsection{NiO}
\label{\detokenize{Examples/Examples:nio}}
NiO is one of the antiferromagnetic materials. In this example, we show the most stable magnetic spin ordering for NiO
and its electronic configurations (band strucrue and density of states).
\begin{quote}

Optimized structure (/Results/POSCAR\_rlx\_GGA)

\begin{sphinxVerbatim}[commandchars=\\\{\}]
relaxed poscar
1.000000000
    1.47786935879    0.853248272122    4.82278497551
    \PYGZhy{}1.47786935879    0.853248272122    4.82278497551
    0.0    \PYGZhy{}1.70649654425    4.82278497551
    Ni    O
    2    2
Selective dynamics
Direct
    0.5    0.5    0.5  T  T  T ! Ni1\PYGZus{}up
    \PYGZhy{}0.0    \PYGZhy{}0.0    0.0  T  T  T ! Ni1\PYGZus{}down
    0.750000037602    0.750000037602    0.750000037602  T  T  T ! O1
    0.249999962398    0.249999962398    0.249999962398  T  T  T ! O1
\end{sphinxVerbatim}

Band gap (/Results/band\_gap\_GGA.log)

\begin{sphinxVerbatim}[commandchars=\\\{\}]
\PYG{n}{Band} \PYG{n}{gap}\PYG{p}{:}      \PYG{l+m+mf}{3.433} \PYG{n}{eV} \PYG{p}{(}\PYG{n}{Indirect}\PYG{p}{)}

\PYG{n}{VBM}\PYG{p}{:} \PYG{l+m+mf}{0.5}  \PYG{l+m+mf}{0.5}  \PYG{l+m+mf}{0.5}   \PYG{p}{:}      \PYG{l+m+mf}{6.242} \PYG{n}{eV}
\PYG{n}{CBM}\PYG{p}{:} \PYG{l+m+mf}{0.0}  \PYG{l+m+mf}{0.0}  \PYG{l+m+mf}{0.0}   \PYG{p}{:}      \PYG{l+m+mf}{9.675} \PYG{n}{eV}

\PYG{n}{nVBM}\PYG{p}{:} \PYG{l+m+mi}{16}  \PYG{n}{spin}\PYG{p}{:} \PYG{l+m+mi}{1}
\PYG{n}{nCBM}\PYG{p}{:} \PYG{l+m+mi}{17}  \PYG{n}{spin}\PYG{p}{:} \PYG{l+m+mi}{2}
\end{sphinxVerbatim}

Band structure (/Results/band\_GGA.png and /Results/band\_GGA.pdf)
\begin{quote}

\noindent\sphinxincludegraphics[width=300\sphinxpxdimen]{{band_GGA2}.png}
\end{quote}

Density of states (/Results/dos\_GGA.log)
\begin{quote}

\noindent\sphinxincludegraphics[width=150\sphinxpxdimen]{{dos_GGA1}.png}
\end{quote}
\end{quote}



\renewcommand{\indexname}{Index}
\printindex
\end{document}